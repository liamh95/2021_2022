\documentclass[11pt,letterpaper]{article}
\usepackage{amssymb,amsfonts,color,graphicx,amsmath,enumerate}
\usepackage{tikz}
\usepackage{amsthm}

\newcommand{\naturals}{\mathbb{N}}
\newcommand{\integers}{\mathbb{Z}}
\newcommand{\complex}{\mathbb{C}}
\newcommand{\reals}{\mathbb{R}}
\newcommand{\mcal}[1]{\mathcal{#1}}
\newcommand{\rationals}{\mathbb{Q}}
\newcommand{\Lp}[2]{\left\|{#1}\right\|_{L^{#2}}}
\newcommand{\field}{\mathbb{F}}
\newcommand{\affine}{\mathbb{A}}
\newcommand{\E}{\mathbb{E}}
\newcommand{\Prob}{\mathbb{P}}
\newcommand{\Var}{\text{Var}}
\newcommand{\ind}{\mathbbm{1}}
\newcommand{\Cov}{\text{Cov}}

\newenvironment{solution}
{\begin{proof}[Solution]}
{\end{proof}}

\voffset=-3cm
\hoffset=-2.25cm
\textheight=24cm
\textwidth=17.25cm
\addtolength{\jot}{8pt}
\linespread{1.3}

\begin{document}
\begin{center}
{\bf \Large Math 173A - Modular Exponentiation}
\vspace{0.2cm}
\hrule
\end{center}

\begin{enumerate}

    \item Let $p$ be a prime number. Prove that $ord_p$ has the following properties.
    \begin{enumerate}
        \item $ord_p(ab) = ord_p(a) + ord_p(b)$
        \item $ord_p(a + b) \geq \min\{ord_p(a),\ ord_p(b)\}$.
        \item If $ord_p(a)\neq ord_p(b)$, then $ord_p(a+b) = \min\{ord_p(a),\ ord_p(b)\}$.

    \end{enumerate}

    \item Let $p$ be a prime. Show that the only solutions to $x^2 \equiv 1\pmod p$ are $x \equiv \pm 1\pmod p$. Is it important that $p$ is a prime?

    \item \begin{enumerate}
        \item Let $p$ be a prime. Show that if $p\nmid a$, then $a^{p-2}$ is congruent to the multiplicative inverse of $a$ modulo $p$.

        \item Find $17^{-1}$ modulo 101 using the extended Euclidean algorithm and by computing $17^{99}\pmod {101}$ using the square-and-multiply algorithm.
    \end{enumerate}


    \item \begin{enumerate}
        \item Estimate how many multiplication operations modulo $N$ it takes to compute $g^A\pmod N$ using the square-and-multiply algorithm. Computing $a\cdot b\pmod N$ given $a$ and $b$ is one multiplication operation.

        \item Recall that in order to compute $g^A\pmod N$, the square and multiply algorithm computes (and stores) each of the numbers
        \[
            g,\ g^2,\ g^{2^2},\ \ldots,\ g^{2^r},
        \]
        modulo $N$, where $A = A_0 + A_1\cdot 2 + A_2\cdot 2^2 + \cdots + A_r\cdot 2^r$.
    \end{enumerate}

\end{enumerate}

\end{document}