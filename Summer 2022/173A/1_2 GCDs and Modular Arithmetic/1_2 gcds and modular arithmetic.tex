\documentclass[11pt,letterpaper]{article}
\usepackage{amssymb,amsfonts,color,graphicx,amsmath,enumerate}
\usepackage{tikz}
\usepackage{amsthm}

\newcommand{\naturals}{\mathbb{N}}
\newcommand{\integers}{\mathbb{Z}}
\newcommand{\complex}{\mathbb{C}}
\newcommand{\reals}{\mathbb{R}}
\newcommand{\mcal}[1]{\mathcal{#1}}
\newcommand{\rationals}{\mathbb{Q}}
\newcommand{\Lp}[2]{\left\|{#1}\right\|_{L^{#2}}}
\newcommand{\field}{\mathbb{F}}
\newcommand{\affine}{\mathbb{A}}
\newcommand{\E}{\mathbb{E}}
\newcommand{\Prob}{\mathbb{P}}
\newcommand{\Var}{\text{Var}}
\newcommand{\ind}{\mathbbm{1}}
\newcommand{\Cov}{\text{Cov}}

\newenvironment{solution}
{\begin{proof}[Solution]}
{\end{proof}}

\voffset=-3cm
\hoffset=-2.25cm
\textheight=24cm
\textwidth=17.25cm
\addtolength{\jot}{8pt}
\linespread{1.3}

\begin{document}
\begin{center}
{\bf \Large Math 173A - Greatest Common Divisors and Modular Arithmetic}
\vspace{0.2cm}
\hrule
\end{center}

\begin{enumerate}

    \item Prove that if $p$ is a prime number and $p\mid a_1a_2\ldots a_k$ for integers $a_1, \ldots, a_k$, then $p$ divides at least one of the $a_i$'s.


    \item In this exercise we'll prove that prime factorization is unique, i.e. that any integer $a$ may be written
    \begin{equation}\label{factorization}
        a = p_1^{e_1}p_2^{e_2}\cdots p_r^{e_r}
    \end{equation}
    for primes $p_1, \ldots, p_r$ and nonnegative integers $e_1, \ldots, e_r$ where the $p_i$ are unique up to rearrangement.

    A common way to prove that some way of doing something is unique is to do it in two ways and then argue that they're the same.
    To this end, suppose we could write
    \begin{equation}\label{different}
        a = p_1p_2\cdots p_s = q_1q_2\cdots q_t,
    \end{equation}
    where the $p_i$ and $q_j$ are all primes, not necessarily distinct and $s$ may be different from $t$.
    \begin{enumerate}
        \item Where did the $e_i$'s go when we moved from (\ref{factorization}) to (\ref{different})?


        \item Argue that $p_1$ must divide $q_1\cdots q_t$. Use this to conclude that $p_1$ is equal to one of the $q_i$'s. Now reorder so that this is $q_1$.

        \item Now divide both sides of (\ref{different}) by $p_1$. What are you left with? Repeat this argument.

        \item Put the previous pieces together into a short, well-written proof for the uniqueness of prime factorization.
    \end{enumerate}
    

    \item Prove that there are infinitely many prime numbers. \emph{Hint: what if there were only finitely many?}

    \item Compute $\gcd(291, 252)$ and find integers $u$ and $v$ such that
    \[
        291u + 252v = \gcd(291, 252).
    \]

    \item 
\end{enumerate}

\end{document}