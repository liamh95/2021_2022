\documentclass[11pt,letterpaper]{article}
\usepackage{amssymb,amsfonts,color,graphicx,amsmath,enumerate}
\usepackage{tikz}
\usepackage{amsthm}

\newcommand{\naturals}{\mathbb{N}}
\newcommand{\integers}{\mathbb{Z}}
\newcommand{\complex}{\mathbb{C}}
\newcommand{\reals}{\mathbb{R}}
\newcommand{\mcal}[1]{\mathcal{#1}}
\newcommand{\rationals}{\mathbb{Q}}
\newcommand{\Lp}[2]{\left\|{#1}\right\|_{L^{#2}}}
\newcommand{\F}{\mathbb{F}}
\newcommand{\affine}{\mathbb{A}}
\newcommand{\E}{\mathbb{E}}
\newcommand{\Prob}{\mathbb{P}}
\newcommand{\Var}{\text{Var}}
\newcommand{\ind}{\mathbbm{1}}
\newcommand{\Cov}{\text{Cov}}

\newenvironment{solution}
{\begin{proof}[Solution]}
{\end{proof}}

\voffset=-3cm
\hoffset=-2.25cm
\textheight=24cm
\textwidth=17.25cm
\addtolength{\jot}{8pt}
\linespread{1.3}

\begin{document}
\begin{center}
{\bf \Large Math 173A - Homework 3}
\vspace{0.2cm}
\hrule
\end{center}


% hill cipher
% existence of primitive roots modulo p
% compute the parity of discrete log without actually computing it (look at notes)
% man in the middle DH

\begin{enumerate}

    \item Do the following exercises from the textbook. 2.17, 2.18a and d, 2.21, 2.25, 2.27.

    \item Let $p = 601$, which is prime.
    \begin{enumerate}
        \item Show that if an integer $r<600$ divides $600$, then it divides at least oone of 300, 200, 120 (these numbers are 600/2, 600/3, and 600/5).

        \item Show that if the order of 7 in $\F_{601}$ is less than 600, then it divides one of the numbers 300, 200, 120.

        \item A calculation shows that
        \[
            7^{300}\equiv 600,\quad 7^{200}\equiv 576\quad 7^{120}\equiv 423\pmod{601}.
        \]
        Why can we conclude that the order of 7 does not divide 300, 200, or 120?

        \item Show that 7 is a primitive root in $\F_{601}$.

        \item In general, suppose $p$ is a prime and $p-1 = q_1^{e_1}q_2^{e_2}\cdots q_s^{e_s}$ is the factorization of $p-1$ into primes. Describe a procedure to check whether a number $g$ is a primitive root mod $p$.
    \end{enumerate}

    \item Let $p \equiv 3\pmod 4$ be a prime.
    Show that $x^2 \equiv -1\pmod p$ has no solutions.
    \emph{Hint: Suppose a solution $x$ exists. Raise both sides to the power $(p-1)/2$.}

\end{enumerate}

\end{document}