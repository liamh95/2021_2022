\documentclass[11pt,letterpaper]{article}
\usepackage{amssymb,amsfonts,color,graphicx,amsmath,enumerate}
\usepackage{tikz}
\usepackage{amsthm}

\newcommand{\naturals}{\mathbb{N}}
\newcommand{\integers}{\mathbb{Z}}
\newcommand{\complex}{\mathbb{C}}
\newcommand{\reals}{\mathbb{R}}
\newcommand{\mcal}[1]{\mathcal{#1}}
\newcommand{\rationals}{\mathbb{Q}}
\newcommand{\Lp}[2]{\left\|{#1}\right\|_{L^{#2}}}
\newcommand{\F}{\mathbb{F}}
\newcommand{\affine}{\mathbb{A}}
\newcommand{\E}{\mathbb{E}}
\newcommand{\Prob}{\mathbb{P}}
\newcommand{\Var}{\text{Var}}
\newcommand{\ind}{\mathbbm{1}}
\newcommand{\Cov}{\text{Cov}}

\newenvironment{solution}
{\begin{proof}[Solution]}
{\end{proof}}

\voffset=-3cm
\hoffset=-2.25cm
\textheight=24cm
\textwidth=17.25cm
\addtolength{\jot}{8pt}
\linespread{1.3}

\begin{document}
\begin{center}
{\bf \Large Math 173A - RSA and Primality Testing}
\vspace{0.2cm}
\hrule
\end{center}

\begin{enumerate}

    \item Why should you choose your public exponent to be 1 or 2 in RSA?

    \vfill

    \item Suppose $n = pqr$ is the product of three distinct primes.
    How would an RSA-type scheme work in this case?
    In particular, what relation would the encryption and decryption exponents $e$ and $d$ satisfy?

    \vfill

    \item The number 561 factors as $3\cdot 11\cdot 17$.
    First use Fermat's little theorem to show that
    \[
        a^{561}\equiv a\pmod 3,\quad a^{561}\equiv a\pmod {11},\quad a^{561}\equiv a\pmod {17},
    \]
    for every value of $a$.
    Explain why these three congruences imply that $a^{561}\equiv a\pmod{561}$ for all $a$.
    \vfill
\end{enumerate}

\end{document}