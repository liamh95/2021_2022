\documentclass[11pt,letterpaper]{article}
\usepackage{amssymb,amsfonts,color,graphicx,amsmath,enumerate}
\usepackage{tikz}
\usepackage{amsthm}

\newcommand{\naturals}{\mathbb{N}}
\newcommand{\Z}{\mathbb{Z}}
\newcommand{\complex}{\mathbb{C}}
\newcommand{\reals}{\mathbb{R}}
\newcommand{\mcal}[1]{\mathcal{#1}}
\newcommand{\rationals}{\mathbb{Q}}
\newcommand{\Lp}[2]{\left\|{#1}\right\|_{L^{#2}}}
\newcommand{\F}{\mathbb{F}}
\newcommand{\affine}{\mathbb{A}}
\newcommand{\E}{\mathbb{E}}
\newcommand{\Prob}{\mathbb{P}}
\newcommand{\Var}{\text{Var}}
\newcommand{\ind}{\mathbbm{1}}
\newcommand{\Cov}{\text{Cov}}

\newenvironment{solution}
{\begin{proof}[Solution]}
{\end{proof}}

\voffset=-3cm
\hoffset=-2.25cm
\textheight=24cm
\textwidth=17.25cm
\addtolength{\jot}{8pt}
\linespread{1.3}

\begin{document}
\begin{center}
{\bf \Large Math 173A - Discrete Logarithms}
\vspace{0.2cm}
\hrule
\end{center}

\begin{enumerate}

    \item Let $g$ be a primitive root for $\F_p$.
    \begin{enumerate}
        \item Suppose that $x = a$ and $x = b$ are both integer solutions to the congruence $g^x\equiv h\pmod p$.
        Prove that $a\equiv b\pmod {p-1}$.
        Explain why this means the map
        \begin{align*}
            \log_g: \F_p^\times &\to \Z/(p-1)\Z\\
            g^x &\mapsto x\pmod {p-1}
        \end{align*}
        is well-defined.

        \item Prove that $\log_g(h_1h_2) \equiv \log_g(h_1) + \log_g(h_2)\pmod{p-1}$ for all $h_1, h_2\in \F_p^\times$.

        \item Prove that $\log_g(h^n) \equiv n\log_g(h)$ for all $h\in \F_p^\times$ and $n\in \Z$.
    \end{enumerate}

    % \item
    % \begin{enumerate}
    %     \item Compute $6^5\pmod {11}$ using the square-and-multiply algorithm.

    %     \item Given that 2 is a primitive root modulo 11, consider the equation $2^x\equiv 6\pmod{11}$.
    %     \emph{Without finding $x$}, determine whether $x$ is even odd. (Remember that the discrete logarithm is defined modulo $p-1$. Why does it make sense to ask whether $x$ is even or odd?)
    % \end{enumerate}

    \item This exercise describes a public key cryptosystem that requires Alice and Bob to exchange several messages.
    We illustrate it with an example using small numbers.

    Alice and Bob publicly agree on a prime $p = 23$.
    Suppose Alice wants to send Bob the message $m = 11$.
    She chooses a random exponent $a = 3$ and sends the number $u \equiv m^a \equiv 20\pmod{23}$ to Bob.
    Bob chooses a random exponent $b = 9$ and sends $v \equiv u^b \equiv 5\pmod{23}$ back to Alice.
    Alice then computes $w \equiv v^{15} \equiv 19\pmod{23}$ and sends $w$ to Bob.
    Finally, Bob computes $w^5 \equiv 11\pmod{23}$ to recover Alice's original message.

    \begin{enumerate}
        \item Explain why this system works.
        In particular, how are Alice's exponents $a = 3$ and 15 related?
        Likewise, how are Bob's exponents $b = 9$ and 5 related?

        \item How would you explain this cryptosystem in general? That is, how does it work outside of just this example?

        \item Can you break this system if you can solve the discrete logarithm problem?
    \end{enumerate}

\end{enumerate}

\end{document}