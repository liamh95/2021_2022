\documentclass[12pt]{article}  
%%Read the manual for other options. 

\pagestyle{empty} %%Eliminates page numbers
%%\input rmb_macros
%%Collect your favorite macros in a 
%%separate file

%\input amssym.def
%\input amssym
%\input mssymb
%%Defines additional symbols



\usepackage{graphics}
\usepackage{amsmath,amssymb,amsthm, multicol}
\usepackage[pdftex]{graphicx}
\usepackage{epsf}
%%Use to include pictures. 

%\newcommand{\comment}[1]{}
%\newcommand{\sobolev}[2]{W^{#1,#2}}
%\newcommand{\sobolev}[2]{L^#2_#1}
%%Some examples of macros or new commands.
\newcommand{\integers}{\mathbb{Z}}

\addtolength{\oddsidemargin}{-.75in}
\addtolength{\evensidemargin}{-.75in}
\addtolength{\textwidth}{1.5in}
\addtolength{\topmargin}{-1in}
\addtolength{\textheight}{2.25in}
%%Set margins, defaults are ok. 

\begin{document}
\begin{flushleft} 
%%Paragraphs will not be indented in this 
%%environment
\centerline{\LARGE{Week 2 Quiz}} 
\vspace{5 mm}
{Student ID Number:}\hfill  
%%\hfill forces following text 
%%to right margin
{Name \rule {2 in}{0.01in}}\\
Math 173A, 11AM
\\
%%gives a line of length 2in and 
%%thickness 0.01in
{Please justify all your answers}\hfill {July 1, 2022}
\\
{Please also write your full name on the back} 

\medskip
\end{flushleft}

\begin{enumerate}
    \item Fill in the blank or answer with ``True'' or ``False''.
        \begin{enumerate}
            \item Fix a prime $p$ and suppose that $a$ is coprime to $p$. The smallest positive integer $k$ such that $a^k \equiv 1\pmod{p}$ is called the \rule{2.5cm}{.15mm} of $a \pmod{p}$.
            \item True or false? If $p$ is prime and $a$ is any integer then $a^{p-1}\equiv 1 \pmod{p}$.
            \item Fix a prime $p$. An element $g \in \mathbb{F}_p^\times$ whose powers give every element of $\mathbb{F}_p^\times$ is called a \rule{2.5cm}{.15mm} of $\mathbb{F}_p^\times$.
        \end{enumerate}
    \vfill
    \item 
    \begin{enumerate}
        \item Solve $7d\equiv 1\pmod{30}$.
        \vfill
        \item Suppose you write a message as a number $m\pmod{31}$. Encrypt $m$ as $m^7\pmod{31}$. How would you decrypt? \textit{Hint: Decryption can be done by raising the ciphertext to a power mod 31. Fermat's little theorem will be useful.}
        \vfill
    \end{enumerate}

    \item \begin{enumerate}
        \item What is one topic or example from the course so far that you find confusing? What do you find confusing about it?

        \vfill

        \item What would you like to see more of in the discussion sections? What would you like to see less of?
        \vfill

    \end{enumerate}
\end{enumerate}

%\vfill will divide page evenly
%use \begin{enumerate} environment for ordered lists
\end{document}