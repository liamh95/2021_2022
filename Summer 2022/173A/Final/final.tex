\documentclass[11pt]{article}
\usepackage{amsmath, amssymb, amsthm, verbatim,enumerate,bbm, mathtools}
\usepackage{indentfirst}



\date{\today}
\parindent 5mm
\parskip 0.2mm
\oddsidemargin  0pt \evensidemargin 0pt \marginparwidth 0pt
\marginparsep 0pt \topmargin 0pt \headsep 0pt \textheight 8.8in
\textwidth 6.6in
\renewcommand{\baselinestretch}{1.03}

\allowdisplaybreaks

\theoremstyle{plain}
\newtheorem{theorem}{Theorem}[section]
\newtheorem{lemma}[theorem]{Lemma}
\newtheorem{claim}[theorem]{Claim}
\newtheorem{proposition}[theorem]{Proposition}
\newtheorem{observation}[theorem]{Observation}
\newtheorem{corollary}[theorem]{Corollary}
\newtheorem{conjecture}[theorem]{Conjecture}
\newtheorem{problem}[theorem]{Problem}
\newtheorem{remark}[theorem]{Remark}
\newtheorem{definition}[theorem]{Definition}
\newtheorem{property}[theorem]{Property}
\newcommand{\Bin}{\ensuremath{\textrm{Bin}}}
\newcommand{\Var}{\text{Var}}
\newcommand{\F}{\mathbb{F}}
\newcommand{\N}{\mathbb{N}}
\newcommand{\R}{\mathbb{R}}

\title{Final Exam -- Math 173A}
\author{Instructor: Liam Hardiman}
\date{\today}

\begin{document}

\maketitle


{\bf Instructions: } \begin{itemize} 
\item You must show your work and clearly
explain your line of reasoning.
\item You may use a calculator, but only for basic arithmetic.
If you didn't bring a calculator, you may use your phone, but it must be set to airplane/flight mode and you must clear all notifications before you start.
\item You may use one sheet (front and back) of handwritten notes (or a printed sheet of digitally handwritten notes).
\item You have 90 minutes to complete the exam.
\end{itemize}

\medskip
\medskip

\center{{\bf GOOD LUCK!}}

\newpage
\ 
\newpage

%LSB attack bonus?

\begin{enumerate}
    \item (10 points)

    \begin{enumerate}

        \item (4 points) Show that if $x^2 \equiv y^2 \pmod n$ and $x\not\equiv \pm y\pmod n$, then $\gcd(x+y, n)$ is a nontrivial factor of $n$

        \vfill

        \item (6 points) Let $N = pq$ be a product of two large, distinct, and unknown primes.
        Suppose you have access to an oracle that accepts as input an integer $a$ and returns an integer $b$ such that $b^2 \equiv a\pmod N$.
        If no such $b$ exists, the oracle returns the symbol $\perp$.

        Explain how you can use this oracle to factor $N$.
        You may use results from your homework or lecture about square roots modulo $N$ without proof (be sure to state them clearly, though).

        \vfill

    \end{enumerate}

    \newpage
    \ 
    \newpage

    \item (10 points)
    \begin{enumerate}
        \item (4 points) Show that the RSA encryption scheme that we've studied is \emph{homomorphic}.
        That is, show that if we know the encryptions of messages $m_1$ and $m_2$, then we can easily obtain the encryption of the message $m_1m_2$ \emph{without learning the private key.}
        \vfill

        \item (6 points) Bob uses RSA to receive a single ciphertext $c$ corresponding to the message $m$.
        His public modulus is $N$ and his public encryption exponent is $e$.
        Bob agrees to decrypt any ciphertext Eve sends to him and show her the result as long as Eve does not send him $c$.
        Eve sends Bob the ciphertext $2^ec\pmod N$.
        Show how Eve can use this to find $m$.
        \vfill
    \end{enumerate}

    \newpage
    \ 
    \newpage

    \item (10 points) 
    \begin{enumerate}
        \item (4 points) What kinds of numbers are Pollard's $p-1$ algorithm particularly good at factoring? Why?

        \vfill
        \item (6 points) Use Pollard's $p-1$ algorithm to factor 1649.

        \vfill
    \end{enumerate}


    \newpage
    \ 
    \newpage

    \item (10 points)
    Samantha sets up the parameters so she can sign documents with the ElGamal signature scheme.
    She publishes a large prime $p$, a primitive root $g$ modulo $p$, and a verification key $A\equiv g^a\pmod p$ corresponding to her private signing key $1\leq a\leq p-1$.

    Eve chooses $u$ and $v$ such that $\gcd(v, p-1) = 1$ and computes
    \[
        S_1 \equiv A^vg^u\pmod p\qquad\text{and}\qquad S_2\equiv -S_1v^{-1}\pmod{p-1}.
    \]
    \begin{enumerate}
        \item (7 points)
        Show that the pair $(S_1, S_2)$ is a valid signature for the document $D \equiv S_2u\pmod{p-1}$.
        Of course, it's likely that $D$ is a meaningless document.

        \vfill

        \item (3 points)
        Recall that a \emph{hash function} takes in a document of arbitrary size and returns an integer of a fixed size (assume it returns an integer modulo $p-1$ for this problem).
        This function is difficult to invert.

        Suppose a hash function $h$ is used so that Samantha's signature must be valid for $h(D)$ instead of for the document $D$ itself.
        Explain how this protects against the forgery outlined in part (a).

        \vfill
    \end{enumerate}

    \newpage
    \ 
    \newpage

    \item (10 points)
    Alice chooses two large primes $p$ and $q$.
    She chooses an integer $e$ relatively prime to $(p-1)(q-1)$ and computes $d$ such that $de\equiv 1\pmod{(p-1)(q-1)}$.
    Her RSA public key is $N = pq$ and $e$.

    Suppose Alice also computes the following values
    \[
        d_p \equiv d\pmod{p-1}\qquad\text{and}\qquad d_q \equiv d\pmod{q-1}.
    \]

    \begin{enumerate}
        \item (7 points) Bob chooses a message $m$ and sends Alice the ciphertext $c \equiv m^e\pmod N$.
        Instead of computing $c^d\pmod N$ to recover $m$, Alice instead computes
        \[
            m_1 \equiv c^{d_p}\pmod p\qquad\text{and}\qquad m_2\equiv c^{d_q}\pmod q.
        \]

        Explain how she can use $m_1$ and $m_2$ to recover $m$.

        \vfill

        \item (3 points) Do you think this method is more or less efficient than just computing $c^d\pmod N$?
        Why?

        \vfill
    \end{enumerate}

    \newpage
    \ 
    \newpage

    \item (10 points)
    Let $p$ be a prime and let $g$ be an integer.
    The \emph{Decision Diffie-Hellman Problem} is as follows.
    Suppose that you are given three numbers $A$, $B$, and $C$, and suppose that $A$ and $B$ are equal to
    \[
        A\equiv g^a\pmod p\qquad\text{and}\qquad B\equiv g^b\pmod p,
    \]
    but that you don't necessarily know the values of the exponents $a$ and $b$.
    Determine whether $C$ is equal to $g^{ab}\pmod p$.

    Prove that an algorithm that solves the Diffie-Hellman problem can be used to solve the decision Diffie-Hellman problem.
    \newpage
    \ 


\end{enumerate}


\end{document}
