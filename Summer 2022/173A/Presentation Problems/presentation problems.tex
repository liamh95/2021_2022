\documentclass[11pt,letterpaper]{article}
\usepackage{amssymb,amsfonts,color,graphicx,amsmath,enumerate}
\usepackage{tikz}
\usepackage{amsthm}

\newcommand{\naturals}{\mathbb{N}}
\newcommand{\Z}{\mathbb{Z}}
\newcommand{\complex}{\mathbb{C}}
\newcommand{\reals}{\mathbb{R}}
\newcommand{\mcal}[1]{\mathcal{#1}}
\newcommand{\rationals}{\mathbb{Q}}
\newcommand{\Lp}[2]{\left\|{#1}\right\|_{L^{#2}}}
\newcommand{\F}{\mathbb{F}}
\newcommand{\affine}{\mathbb{A}}
\newcommand{\E}{\mathbb{E}}
\newcommand{\Prob}{\mathbb{P}}
\newcommand{\Var}{\text{Var}}
\newcommand{\ind}{\mathbbm{1}}
\newcommand{\Cov}{\text{Cov}}
\newcommand{\lcm}{\text{lcm}}

\newenvironment{solution}
{\begin{proof}[Solution]}
{\end{proof}}

\voffset=-3cm
\hoffset=-2.25cm
\textheight=24cm
\textwidth=17.25cm
\addtolength{\jot}{8pt}
\linespread{1.3}

\begin{document}
\begin{center}
{\bf \Large Math 173A - Presentation Problem Set}
\vspace{0.2cm}
\hrule
\end{center}


% hill cipher
% existence of primitive roots modulo p
% compute the parity of discrete log without actually computing it (look at notes)
% man in the middle DH

\section{Basic Number Theory}
\begin{enumerate}
    \item Let $F_1 = 1$, $F_2 = 1$, $F_{n+1} = F_n + F_{n-1}$ define the Fibonacci numbers 1, 1, 2, 3, 5, 8, $\ldots$.
    Use the Euclidean algorithm to compute $\gcd(F_n, F_{n-1})$ for all $n\geq 1$.

    \item Find the last 2 digits of $123^{562}$.

    \item Here is how to construct the $x$ guaranteed by the general form of the Chinese remainder theorem.
    Suppose $m_1,\ m_2,\ \ldots,\ m_k$ are integers with $\gcd(m_i, m_j) = 1$ whenever $i\neq j$.
    Let $a_1,\ a_2,\ \ldots,\ a_k$ be integers.
    Perform the following procedure.
    \begin{enumerate}[(i)]
        \item For $i= 1,\ldots, k$ and let $z_i = m_1m_2\cdots m_{i-1}m_{i+1}\cdots m_k$.
        \item For $i= 1, \ldots, k$ and let $y_i \equiv z_i^{-1}\pmod m_i$.
        \item Let $x = a_1y_1z_1 + \cdots + a_ky_kz_k$.
    \end{enumerate}
    Show that $x\equiv a_i\pmod{m_i}$ for all $i$.

    \item Let $p$ be an odd prime and let $a$ be an integer that is not divisible by $p$, and let $b$ be a square root of $a$ modulo $p$.
    \begin{enumerate}
        \item Prove that for some choice of $k$, the number $b+kp$ is a square root of $a$ modulo $p^2$.
        \item Suppose that $b$ i sa square root of $a$ modulo $p^n$.
        Prove that for some choice of $j$, the number $b+jp^n$ is a square root of $a$ modulo $p^{n+1}$.
        \item Show that if $p$ is an odd prime and if $a$ has a square root modulo $p$, then $a$ has a square root modulo $p^n$ for every power of $p$.
        Is this true if $p=2$?
    \end{enumerate}

    \item Let $n=pq$ with $p$ and $q$ distinct odd primes.
    \begin{enumerate}
        \item Suppose that $\gcd(a, pq) = 1$.
        Prove that if the equation $x^2 \equiv a\pmod n$ has any solutions, then it has four solutions.

        \item Suppose that you had a machine that could find all four solutions for some given $n$.
        How could you use this machine to factor $n$?
    \end{enumerate}

    \item Let $a,b,m,n$ be integers with $\gcd(m,n) = 1$.
    Let
    \[
        c \equiv(b-a)\cdot m^{-1}\pmod n.
    \]
    Prove that $x = a+cm$ is a solution to
    \[
        x\equiv a\pmod m\qquad \text{and}\qquad x\equiv b\pmod n
    \]
    and that every solution to these simultaneous congruences has the form $x=a+cm+ymn$ for some $y\in \Z$.
\end{enumerate}










\section{Euler's $\phi$ function}
\begin{enumerate}
    \item For any two integers $m$ and $n$, prove that $\phi(\lcm(m,n))\cdot \phi(\gcd(m,n)) = \phi(m)\phi(n)$.

    \item Let $p_1,\ p_2,\ \ldots,\ p_r$ be the distinct primes that divide $N$.
    Prove that
    \[
        \phi(N) = N\prod_{i=1}^r\left(1 - \frac{1}{p_i}\right).
    \]

    \item Let $N$, $c$, and $e$ be positive integers satisfying the conditions $\gcd(N, c) = 1$ and $\gcd(e, \phi(N)) = 1$.
    Explain how to solve the congruence
    \[
        x^e\equiv c\pmod N,
    \]
    assuming you know the value of $\phi(N)$.

    \item Find all $n$ such that $\phi(n)$ is odd and prove that you have found all such $n$.

    \item For any positive integer $N$, prove that
    \[
        \sum_{d\mid N}\phi(d) = N,
    \]
    where the sum is over all positive divisors of $N$.

    \item For any two integers $m$ and $n$ prove that
    \[
        \phi(mn) = \phi(m)\phi(n)\cdot \frac{\gcd(m,n)}{\phi(\gcd(m,n))}.
    \]

    \item Let $M$ and $N$ be integers satisfying $\gcd(M, N) = 1$.
    Prove the multiplication formula
    \[
        \phi(MN) = \phi(M)\phi(N).
    \]

    \item Prove that if $n$ has $r$ distinct odd prime factors, then $2r\mid \phi(n)$.


\end{enumerate}










\section{Abstract Algebra}
\begin{enumerate}
    \item Show that the set of all invertible $2\times 2$ matrices with entries in $\F_p$ is a noncommutative group for every prime $p$.

    \item Let $G$ be a group and let $d\geq 1$ be an integer. Define a subset of $G$ by
    \[
        G[d] = \{g\in G: g^d = e\}.
    \]
    \begin{enumerate}
        \item Prove that if $g$ is in $G[d]$, then $g^{-1}$ is in $G[d]$.
        \item Suppose that $G$ is commutative. Prove that if $g_1$ and $g_2$ are in $G[d]$, then $g_1g_2$ is in $G$ as well.
        \item Deduce that if $G$ is commutative, then $G[d]$ is a group.
        \item Show by an example that if $G$ is not a commutative group, then $G[d]$ need not be a group.
    \end{enumerate}

    \item If $\F$ is a field, we define the \emph{characteristic} of $\F$ to be the smallest positive integer $k$ such that
    \[
        \underbrace{1 + 1 + \cdots + 1}_{k\text{ times}} = 0.
    \]
    If there is no such $k$, we say that $\F$ has characteristic zero.
    \begin{enumerate}
        \item Give an example of a field with characteristic zero.

        \item Show that if $\F$ is a field with positive characteristic, the characteristic must be a prime number.
    \end{enumerate}

    \item If $m$ is composite, why is $\Z/m\Z$ not a field?
\end{enumerate}










\section{Discrete Logarithms}
\begin{enumerate}
    \item Let $p$ be prime.
    \begin{enumerate}
        \item Let $q$ be a prime number such that $q\mid p-1$.
        Prove that $\F_p^\times$ has an element of order $q$.

        \item Let $N$ be an integer such that $N\mid p-1$.
        Prove that $\F_p^\times$ has an element of order $N$.
    \end{enumerate}

    \item It can be shown that 5 is a primitive root for the prime 1223.
    You want to solve the discrete logarithm problem $5^x\equiv 3\pmod{1223}$.
    Given that $3^{611} \equiv 1\pmod{1223}$, determine whether $x$ is even or odd.

    \item In the Diffie-Hellman key exchange protocol, Alice and Bob choose a primitive root $g$ for a large prime $p$.
    Alice sends $x_1\equiv g^a\pmod p$ to Bob and Bob sends $x_2 \equiv g^b\pmod p$ to Alice.
    Suppose Eve bribes Bob to tell her the values of $b$ and $x_2$.
    However, Bob forgets to tell Eve the value of $g$.
    Suppose $\gcd(b, p-1) = 1$.
    Show that Eve can determine $g$ from the knowledge of $p$, $x_2$ and $b$.

    \item Use the Pohlig-Hellman algorithm to solve the discrete logarithm problem
    \[
        g^x = a
    \]
    in $\F_p$ where $p = 181$, $g = 2$ and $a = 100$.

    \item In the ElGamal cryptosystem, Alice and Bob use $p=17$ and $g=3$.
    Alice chooses her secret exponent to be $a=6$, so $A\equiv 15\pmod p$.
    Bob sends the ciphertext $(7,6)$.
    Find $m$.

    \item Using the baby-step giant-step algorithm, find $x$ so that $5^x\equiv 193\pmod{503}$.
    You may use a calculator.
\end{enumerate}










\section{RSA}

\begin{enumerate}
    \item Bob uses RSA to receive a single ciphertext $c$ corresponding to the message $m$.
    His public modulus is $N$ and his public encryption exponent is $e$.
    Since Bob feels guilty that his system was used only once, he agrees to decrypt any ciphertext he receives as long as it's not $c$, and return the decryption to that person.
    Eve sends him the ciphertext $2^ec\pmod N$.
    Show how Eve can use this to find $m$.


    \item Suppose you are as system administrator setting up a messaging service for a company.
    You want users to encrypt their communications using RSA, so you generate public and private keys for them.
    You do this by generating a list of 100 huge primes, say with 1024 bits each.
    You then generate each user's public modulus by choosing two primes from this list, being extra careful so that no two users have the same public modulus.

    Explain why if there are more than 50 users on this network, some of the users' communications can be compromised.


    \item Alice and Bob are good friends, so they decide to use the same public RSA modulus, but with different encryption exponents $e$ and $f$.
    Suppose Charlie encrypts the message $m$ and sends it to Alice and Bob, using their respective public keys in both cases.
    Show that if Eve can intercept these ciphertexts then she can recover $m$.

    \item Suppose $n = pq$ is a product of two primes and suppose that $\gcd(a, pq) = 1$.
    \begin{enumerate}
        \item Show that if $x^2 \equiv a\pmod n$ has any solutions in $\Z/n\Z$, then it has exactly four.

        \item If, for some $a\in \Z/n\Z$, you know all four solutions to $x^2\equiv a\pmod n$, show that you can efficiently factor $n$.
    \end{enumerate}


    \item Your opponent uses RSA with $n=pq$ and encryption exponent $e$ and encrypts a message $m$.
    This yields the ciphertext $c\equiv m^p\pmod n$.
    A spy tells you that, for this particular message, $m^{12345}\equiv 1\pmod n$.
    Describe how to determine $m$.
\end{enumerate}










\section{Primality Testing and Factorization}

\begin{enumerate}
    \item \begin{enumerate}
        \item What is a Fermat witness for a composite number?

        \item List all Fermat witnesses for 15.

        \item What is a Miller-Rabin witness?

        \item List all Miller-Rabin witnesses for 15.
    \end{enumerate}

    \item Describe the pollard $p-1$ algorithm and use it to factor $2^9-1$.

    \item Recall that a composite number $n$ is a Carmichael number if $a^n\equiv a \pmod n$.
    Show that all Carmichael numbers are odd.

    \item Let $N = 561$. Show that $a=2$ is a Miller-Rabin witness to the compositeness of $N$ and that this can be used to factor $N$.

    \item Suppose $m$ is a positive integer such that $6m+1$, $12m+1$ and $18m+1$ are all primes.
    Let $n = (6m+1)(12m+1)(18m+1)$.
    Prove that $n$ is a Carmichael number.


\end{enumerate}

\end{document}