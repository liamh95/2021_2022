\documentclass[11pt,letterpaper]{article}
\usepackage{amssymb,amsfonts,color,graphicx,amsmath,enumerate}
\usepackage{tikz}
\usepackage{amsthm}

\newcommand{\naturals}{\mathbb{N}}
\newcommand{\integers}{\mathbb{Z}}
\newcommand{\complex}{\mathbb{C}}
\newcommand{\reals}{\mathbb{R}}
\newcommand{\mcal}[1]{\mathcal{#1}}
\newcommand{\rationals}{\mathbb{Q}}
\newcommand{\Lp}[2]{\left\|{#1}\right\|_{L^{#2}}}
\newcommand{\F}{\mathbb{F}}
\newcommand{\affine}{\mathbb{A}}
\newcommand{\E}{\mathbb{E}}
\newcommand{\Prob}{\mathbb{P}}
\newcommand{\Var}{\text{Var}}
\newcommand{\ind}{\mathbbm{1}}
\newcommand{\Cov}{\text{Cov}}

\newenvironment{solution}
{\begin{proof}[Solution]}
{\end{proof}}

\voffset=-3cm
\hoffset=-2.25cm
\textheight=24cm
\textwidth=17.25cm
\addtolength{\jot}{8pt}
\linespread{1.3}

\begin{document}
\begin{center}
{\bf \Large Math 173A - Homework 4}
\vspace{0.2cm}
\hrule
\end{center}


% side channel attack on RSA
% probabilistic encryption
% RSA with CRT
% Fermat factorization

\begin{enumerate}

\item Do the following book exercises: 3.1(a) and (b), 3.2, 3.10, 3.11, 3.15(a)


% Let us now investigate side-channel attacks against RSA. In a simple imple-
% mentation of RSA without any countermeasures against side-channel leakage, the
% analysis of the current consumption of the microcontroller in the decryption part
% directly yields the private exponent. Figure 7.5 shows the power consumption of an
% implementation of the square-and-multiply algorithm. If the microcontroller com-
% putes a squaring or a multiplication, the power consumption increases. Due to the
% small intervals in between the loops, every iteration can be identified. Furthermore,
% for each round we can identify whether a single squaring (short duration) or a squar-
% ing followed by a multiplication (long duration) is being computed.
\item 
    

\end{enumerate}

\end{document}