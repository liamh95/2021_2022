\documentclass[11pt,letterpaper]{article}
\usepackage{amssymb,amsfonts,color,graphicx,amsmath,enumerate}
\usepackage{tikz}
\usepackage{amsthm}

\newcommand{\naturals}{\mathbb{N}}
\newcommand{\integers}{\mathbb{Z}}
\newcommand{\complex}{\mathbb{C}}
\newcommand{\reals}{\mathbb{R}}
\newcommand{\mcal}[1]{\mathcal{#1}}
\newcommand{\rationals}{\mathbb{Q}}
\newcommand{\Lp}[2]{\left\|{#1}\right\|_{L^{#2}}}
\newcommand{\field}{\mathbb{F}}
\newcommand{\affine}{\mathbb{A}}
\newcommand{\E}{\mathbb{E}}
\newcommand{\Prob}{\mathbb{P}}
\newcommand{\Var}{\text{Var}}
\newcommand{\ind}{\mathbbm{1}}
\newcommand{\Cov}{\text{Cov}}

\newenvironment{solution}
{\begin{proof}[Solution]}
{\end{proof}}

\voffset=-3cm
\hoffset=-2.25cm
\textheight=24cm
\textwidth=17.25cm
\addtolength{\jot}{8pt}
\linespread{1.3}

\begin{document}
\begin{center}
{\bf \Large Math 173A - Homework 2}
\vspace{0.2cm}
\hrule
\end{center}


% hill cipher
% existence of primitive roots modulo p
% compute the parity of discrete log without actually computing it (look at notes)
% man in the middle DH

\begin{enumerate}

    \item Do the following exercises from the textbook. 2.5, 2.6, 2.7, 2.9.

    \item Consider this combination of the Caesar cipher and the multiplication cipher briefly discussed in lecture, known as the \emph{affine shift cipher}.
    Fix a prime $p$.
    The key for an affine cipher consists of two integers $k = (k_1, k_2)$ and encryption is defined by
    \begin{align*}
        e_k(m) &= k_1\cdot m + k_2\pmod p.
    \end{align*}
    \begin{enumerate}
        \item What should the decryption function be?

        \item For a fixed prime $p$, how many valid keys are there?

        \item What are the message and ciphertext spaces?

        \item Assuming that $p$ is public knowledge, explain why the affine cipher is vulnerable to a known-plaintext attack?
        How many plaintext-ciphertext pairs are likely needed to recover they key?
    \end{enumerate}


    \item Let's generalize the affine cipher from the previous exercise.
    Now suppose the plaintext $m$, ciphertext $c$, and the second part of the key $k_2$ are vectors consisting of $n$ numbers modulo $p$.
    The first part of they key $k_1$ is an $n\times n$ matrix whose entries are integers modulo $p$.
    Encryption is defined by
    \[
        e_k(m) = k_1\cdot m + k_2 \pmod p,
    \]
    where $k_1\cdot m$ is the matrix-vector product.
    \begin{enumerate}
        \item What should the decryption function be?

        \item How many valid keys are there? \emph{Hint: are some matrices bad for this? How can you count the good ones?}

        \item Why is this cipher vulnerable to a known-plaintext attack?

        \item Explain how any simple substitution cipher that involves a permutation of the alphabet can be thought of as a special case of this cipher.

    \end{enumerate}

    \item \begin{enumerate}
        \item Compute $6^5\pmod{11}$ using the square-and-multiply algorithm.

        \item Assume that $2$ is a primitive root modulo 11 and suppose that $2^x \equiv 6\pmod{11}$.
        \emph{Without finding the value of $x$,} determine whether $x$ is even or odd.
    \end{enumerate}

    \item Recall that in the Elgamal protocol, every time Bob sends a message to Alice he generates a random exponent $k$.
    Suppose Bob is lazy and decides to use the same value of $k$ for multiple messages.
    Explain why this renders his communications with Alice vulnerable to a known-plaintext attack.

\end{enumerate}

\end{document}