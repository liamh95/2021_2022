\documentclass[11pt,letterpaper]{article}
\usepackage{amssymb,amsfonts,color,graphicx,amsmath,enumerate}
\usepackage{tikz}
\usepackage{amsthm}

\newcommand{\naturals}{\mathbb{N}}
\newcommand{\integers}{\mathbb{Z}}
\newcommand{\complex}{\mathbb{C}}
\newcommand{\reals}{\mathbb{R}}
\newcommand{\mcal}[1]{\mathcal{#1}}
\newcommand{\rationals}{\mathbb{Q}}
\newcommand{\Lp}[2]{\left\|{#1}\right\|_{L^{#2}}}
\newcommand{\F}{\mathbb{F}}
\newcommand{\affine}{\mathbb{A}}
\newcommand{\E}{\mathbb{E}}
\newcommand{\Prob}{\mathbb{P}}
\newcommand{\Var}{\text{Var}}
\newcommand{\ind}{\mathbbm{1}}
\newcommand{\Cov}{\text{Cov}}

\newenvironment{solution}
{\begin{proof}[Solution]}
{\end{proof}}

\voffset=-3cm
\hoffset=-2.25cm
\textheight=24cm
\textwidth=17.25cm
\addtolength{\jot}{8pt}
\linespread{1.3}

\begin{document}
\begin{center}
{\bf \Large Math 173A - RSA}
\vspace{0.2cm}
\hrule
\end{center}

\begin{enumerate}

    \item Recall that decrypting in RSA involves inverting a number $e$ modulo $\phi(N)$, where $N$ is a product of two large (distinct) primes $p$ and $q$.
    The difficulty is that if Eve doesn't know $p$ or $q$, it's (presumably) hard for her to compute $\phi(N)$, and therefore hard for her to invert $e$.

    Explain how Eve can find $p$ and $q$ if she knows $N = pq$ (which we always assume she does) and the sum $p+q$. \textit{Hint: expand $(x-a)(x-b)$.}

    \vfill


    \item The ciphertext 75 was obtained using RSA with $N = 437$ and $e=3$.
    You know that the plaintext is either 8 or 9.
    Determine which it is without factoring $N$.

    \vfill

    \item In order to increase security, Bob chooses $n$ and two encryption exponents $e_1$, $e_2$.
    He asks Alice to encrypt her message $m$ to him by first computing $c_1 \equiv m^{e_1}\pmod N$, then encrypting $c_1$ to get $c_2\equiv c_1^{e_2}\pmod N$.
    Alice then sends $c_2$ to Bob.
    Does this double encryption increase security over single encryption?
    Why or why not?

    \vfill

\end{enumerate}

\end{document}