\documentclass[12pt]{article}
\usepackage[left=0.75in,right=0.75in,top=0.75in,bottom=0.75in,
            footskip=0.25in]{geometry}
\usepackage{graphicx,float,hyperref} 
\usepackage{amsmath,amsthm,amssymb,amsfonts,geometry,mathtools,enumerate,bbm}
\usepackage{algpseudocode}
\usepackage{fancyvrb}
 
\theoremstyle{plain}
\newtheorem{theorem}{Theorem}[section]
\newtheorem{lemma}[theorem]{Lemma}
\newtheorem{proposition}[theorem]{Proposition}
\newtheorem{corollary}[theorem]{Corollary}
\newtheorem{conjecture}[theorem]{Conjecture}
\newtheorem{question}[theorem]{Question}
\newtheorem{property}[theorem]{Property}

\theoremstyle{definition}
\newtheorem{definition}[theorem]{Definition}
\newtheorem{problem}[theorem]{Problem}
\newtheorem{example}[theorem]{Example}
\newtheorem{examples}[theorem]{Examples}


\theoremstyle{remark}
\newtheorem{remark}[theorem]{Remark}
\newtheorem{claim}[theorem]{Claim}
\newtheorem{observation}[theorem]{Observation}
\newtheorem{exercise}[theorem]{Exercise}
\newtheorem{exercises}[theorem]{Exercises}

\newcommand{\Bin}{\ensuremath{\textrm{Bin}}}
 
 
\newcommand{\N}{\mathbb{N}}
\newcommand{\Z}{\mathbb{Z}}
\newcommand{\C}{\mathbb{C}}
\newcommand{\R}{\mathbb{R}}
\newcommand{\Q}{\mathbb{Q}}
\newcommand{\F}{\mathbb{F}}

\DeclarePairedDelimiter{\ceil}{\lceil}{\rceil}
\DeclarePairedDelimiter{\floor}{\lfloor}{\rfloor}
 
% \newenvironment{problem}[2][Problem]{\begin{trivlist}
% \item[\hskip \labelsep {\bfseries #1}\hskip \labelsep {\bfseries #2.}]}{\end{trivlist}}
%If you want to title your bold things something different just make another thing exactly like this but replace "problem" with the name of the thing you want, like theorem or lemma or whatever
 
\begin{document}
 
%\renewcommand{\qedsymbol}{\filledbox}
%Good resources for looking up how to do stuff:
%Binary operators: http://www.access2science.com/latex/Binary.html
%General help: http://en.wikibooks.org/wiki/LaTeX/Mathematics
%Or just google stuff
 
\title{Math 173A}
\author{Liam Hardiman}

\maketitle

\begin{abstract}
    I'm writing these lecture notes for UC Irvine's Math 173A course, taught in the summer of 2022.
    This is a five-ish week course where I plan to get through the first three chapters of Hoffstein, Pipher and Silverman's book \cite{HPS}.
    The class structure consists of a two hour lecture followed by a one hour discussion section three days a week.
    I'm aiming to get through two sections of the book per lecture with a midterm after chapter 2.

\end{abstract}


\tableofcontents


\section{An Introduction to Cryptography}
\subsection{Simple Substitution Ciphers}
One of history's oldest examples of encrypting messages is the \emph{shift cipher}, sometimes called the \emph{Caesar cipher} after Julius Caesar, who allegedly used it to encrypt the orders he'd send to his troops.
To encrypt a message, simply shift each letter of the plaintext forward in the alphabet by three, wrapping around if the shifted letter goes past \texttt{Z}. For example, if the key\footnote{We won't rigorously define what ``plaintext'', ``ciphertext'' or ``key'' mean. You can think of the plaintext as being being the human-readable or usable message (maybe consisting of letters or a number) and the ciphertext as being some unreadable sequence of letters or numbers. Then you can think of the key as being some piece of information that tells you how to convert between plain- and ciphertext.} is \texttt{3} and our plaintext is \texttt{hello, world}, then we have the following ciphertext.

\begin{center}
    \texttt{hello world} $\mapsto$ \texttt{KHOOR ZRUOG}
\end{center}

Conversely, if we know the key is \texttt{3} and we're given the ciphertext \texttt{ZHGQH VGDB}, then we simply shift backwards by 3 to obtain the plaintext.

\begin{center}
    \texttt{ZHGQH VGDB} $\mapsto$ \texttt{wedne sday}
\end{center}

One advantage to the shift cipher is that it's really easy to encrypt and decrypt messages if the key is known.
The main disadvantage is that it's only slightly challenging (more annoying than challenging) for an adversary to decrypt messages even if they don't know the key.
If we use the English alphabet, then there are only 26 possible keys and it doesn't take too long to try them all (a few minutes by hand, a fraction of a second even with bad code).
This trial and error method of trying all possible keys, sometimes called \emph{brute forcing}, works because it's pretty unlikely that decrypting with two different keys will yield two plaintexts that are both readable.
For example, suppose we happen upon the following ciphertext
\begin{center}
    \texttt{XPPEE ZXZCC ZH}.
\end{center}
If we suspect that this ciphertext came from a shift cipher, we can just try all possible un-shifts to get the following possible plaintexts.

\begin{center}
    \begin{tabular}{|c|c||c|c|}
    \hline
    key & plaintext & key & plaintext\\
    \texttt{1} & \texttt{woodd ywybb yg} & \texttt{14} & \texttt{jbbqq ljloo lt}\\
    \texttt{2} & \texttt{vnncc xvxaa xf} & \texttt{15} & \texttt{iaapp kiknn ks}\\
    \texttt{3} & \texttt{ummbb wuwzz we} & \texttt{16} & \texttt{hzzoo jhjmm jr}\\
    \texttt{4} & \texttt{tllaa vtvyy vd} & \texttt{17} & \texttt{gyynn igill iq}\\
    \texttt{5} & \texttt{skkzz usuxx uc} & \texttt{18} & \texttt{fxxmm hfhkk hp}\\
    \texttt{6} & \texttt{rjjyy trtww tb} & \texttt{19} & \texttt{ewwll gegjj go}\\
    \texttt{7} & \texttt{qiixx sqsvv sa} & \texttt{20} & \texttt{dvvkk fdfii fn}\\
    \texttt{8} & \texttt{phhww rpruu rz} & \texttt{21} & \texttt{cuujj ecehh em}\\
    \texttt{9} & \texttt{oggvv qoqtt qy} & \texttt{22} & \texttt{bttii dbdgg dl}\\
    \texttt{10} & \texttt{nffuu pnpss px} & \texttt{23} & \texttt{asshh cacff ck}\\
    \texttt{11} & \texttt{meett omorr ow} & \texttt{24} & \texttt{zrrgg bzbee bj}\\
    \texttt{12} & \texttt{lddss nlnqq nv} & \texttt{25} & \texttt{yqqff ayadd ai}\\
    \texttt{13} & \texttt{kccrr mkmpp mu} & \texttt{ } & \texttt{ }\\
    \hline
    \end{tabular}
\end{center}

The only plaintext here that's even remotely readable is \texttt{meett omorr ow}, corresponding to a key of \texttt{11}.
This process of decrypting a ciphertext without knowing the key in advance is called \emph{cryptanalysis}.

Notice that with a shift cipher, each instance of \texttt{a} encrypts to the same character, and so on.
In this setting, once we know what one character maps to, then we know what all the other characters map to as well.
E.g. if we know that \texttt{m} maps to \texttt{X}, then we know that the cipher shifts each character forward by 11, which immediately tells us that \texttt{a} maps to \texttt{L}, and so on.
A more general \emph{simple substitution cipher} decouples the encryptions of different letters, e.g. each \texttt{a} maps to \texttt{C} and each \texttt{b} maps to \texttt{J}, etc.

\begin{question}
    Explain why this particular substitution cipher is not a shift cipher.
\end{question}

\begin{question}
    How many possible keys are there in a substitution cipher? Hint: think of encryption as a function. What properties should this function have?
\end{question}

What would cryptanalysis of a simple substitution cipher look like?
There are more than $10^{26}$ keys in this case.
If we could try a million keys every second, it would still take more than $10^{13}$ years to try them all, so the brute-force solution is infeasible.
Despite the huge number of possible keys, simple substitution ciphers are often really easy to cryptanalyze in practice with simple \emph{frequency analysis}.
The idea is that if the plaintext is more than a few sentences long, then one might expect to see a lot of \texttt{e}'s, \texttt{t}'s and \texttt{a}'s and not many \texttt{z}'s or \texttt{q}'s.
Consequently, if we look at the frequencies of the letters in the ciphertext, it would be reasonable to guess that the most common ciphertext letters correspond to the most common plaintext letters.

For example, suppose we intercept the following message.

\begin{center}
\begin{BVerbatim}
    LWNSOZ BNWVWB AYBNVB SQWVUO HWDIZW RBBNPB POOUWR PAWXAW
    PBWZWM YPOBNP BBNWJP AWWRZS LWZQJB NVIAXA WPBSAL IBNXWA
    BPIRYR POIWRP QOWAIE NBVBNP BPUSRE BNWVWP AWOIHW OIQWAB
    JPRZBN WFYAVY IBSHNP FFIRWV VBNPBB SVWXYA WBNWVW AIENBV
    ESDWAR UWRBVP AWIRVB IBYBWZ PUSREU WRZWAI DIREBH WIATYV
    BFSLWA VHASUB NWXSRV WRBSHB NWESDW ARWZBN PBLNWR WDWAPR
    JHSAUS HESDWA RUWRBQ WXSUWV ZWVBAY XBIDWS HBNWVW WRZVIB
    IVBNVA IENBSH BNWFWS FOWBSP OBWASA BSPQSO IVNIBP RZBSIR
    VBIBYB WRWLES DWARUW RBOPJI REIBVH SYRZPB ISRSRV YXNFAI
    RXIFOW VPRZSA EPRIKI REIBVF SLWAVI RVYXNH SAUPVB SVWWUU
    SVBOIC WOJBSW HHWXBB NWIAVP HWBJPR ZNPFFI RWVV
\end{BVerbatim}
\end{center}

Let's arrange the letters in the ciphertext by frequency.

\begin{center}
\begin{tabular}{c c c c c c c c c c c}
    \texttt{W} & \texttt{B} & \texttt{R} & \texttt{S} & \texttt{I} & \texttt{V} & \texttt{A} & \texttt{P} & \texttt{N} & \texttt{O} & $\cdots$\\
    76 & 64 & 39 & 36 & 36 & 35 & 34 & 32 & 30 & 16 & $\cdots$
\end{tabular}

\end{center}

The letters in standard English text have the following frequencies.

\begin{center}
\begin{tabular}{c c c c c c c c c c c}
    \texttt{E} & \texttt{T} & \texttt{A} & \texttt{O} & \texttt{N} & \texttt{R} & \texttt{I} & \texttt{S} & \texttt{H} & \texttt{D} & $\cdots$\\
    .131 & .105 & .082 & .080 & .071 & .068 & .064 & .061 & .053 & .038 & $\cdots$
\end{tabular}
\end{center}

Since the letter \texttt{W} appears much more frequently than the other letters in the ciphertext, it tips us off that we might be dealing with a substitution cipher and that an \texttt{e} in the plaintext probably maps to a \texttt{W} in the ciphertext. It's also reasonable to guess that the letters \texttt{B}, \texttt{R}, \texttt{S} and \texttt{I} correspond to the letters \texttt{t}, \texttt{a}, \texttt{o} and \texttt{i} in some order.

Looking at individual letter frequencies lets us get our foot in the door, but it doesn't help us much when it comes to differentiating between letters that appear with roughly the same frequency (like \texttt{R} and \texttt{S} in this ciphertext).
If we think about English text for a bit, we notice that certain pairs of letters, called \emph{bigrams}, appear together more frequently than others (e.g. \texttt{q} is almost always followed by a \texttt{u} and \texttt{th} is a common pair).
Here are a few of the bigram frequencies from our ciphertext

\begin{center}
\begin{tabular}{c | c c c c c c c c c}
    &\texttt{W} & \texttt{B} & \texttt{R} & \texttt{S} & \texttt{I} & \texttt{V} & \texttt{A} & \texttt{P} & \texttt{N}\\
    \hline
    \texttt{W} & 3 & 4 & 12 & 2 & 4 & 10 & 14 & 3 & 1\\
    \texttt{B} & 4 & 4 & 0 & 11 & 5 & 5 & 2 & 4 & 20\\
    \texttt{R} & 5 & 5 & 0 & 1 & 1 & 5 & 0 & 3 & 0\\
    \texttt{S} & 1 & 0 & 5 & 0 & 1 & 3 & 5 & 2 & 0\\
    \texttt{I} & 1 & 8 & 10 & 1 & 0 & 2 & 3 & 0 & 0\\
    \texttt{V} & 8 & 10 & 0 & 0 & 2 & 2 & 0 & 3 & 1\\
    \texttt{A} & 7 & 3 & 4 & 2 & 5 & 4 & 0 & 1 & 0\\
    \texttt{P} & 0 & 8 & 6 & 0 & 1 & 1 & 4 & 0 & 0\\
    \texttt{N} & 14 & 3 & 0 & 1 & 1 & 1 & 0 & 7 & 0\\
\end{tabular}
\end{center}

That is, this table tells us that \texttt{WN} appears once and \texttt{NW} appears 14 times.
In English, the letter \texttt{h} frequently comes before \texttt{e} and rarely comes after it, so it's a safe guess that \texttt{h} maps to \texttt{N} in this particular substitution.
Since \texttt{th} is the most common digram in English and \texttt{BN} is the most common digram in the ciphertext, we guess that \texttt{t} maps to \texttt{B}.
Other features of the English language lead to more educated guesses that lead to a full cryptanalysis of the ciphertext.

\begin{problem}
    Finish decrypting the ciphertext. One place to start is by looking for vowels and noting that some vowels like \texttt{a}, \texttt{i} and \texttt{o} tend to avoid each other.
\end{problem}



\subsection{Divisibility and Greatest Common Divisors}
Some of the most widely-used cryptosystems today make heavy use of abstract algebra and number theory.
Roughly speaking, number theory is concerned with properties of the integers, $\Z$, like divisibility and solutions to equations with integer variables.

\begin{definition}
    Let $a$ and $b$ be integers with $b\neq 0$. We say that $b$ \emph{divides} $a$ if $a=bc$ for some integer $c$, in which case, we write $b\mid a$.
\end{definition}

\begin{example}
    \begin{enumerate}[(a)]
        \item We call the integers divisible by 2 \emph{even} and those that aren't \emph{odd}. Is zero even or odd?

        \item 713 is divisible by 23 since $713 = 23\cdot 31$. The numbers used in everyday cryptographic applications are hundreds or even thousands of digits long.

        \item A number $n$ is divisible by 5 if and only if it ends in a 0 or a 5 (when written in base 10, of course). To see this, write
        \[
            n = d_0 + 10d_1 + 10^2d_2 + \cdots + 10^kd_k,
        \]
        where $k\geq 0$ and $d_i \in \{0, 1, 2, \ldots, 9\}$ for all $i$.
        Then $d_0$ is the number that $n$ ``ends'' with, so if it's 0 or 5, we can just factor a 5 out of the right-hand side to see that $n$ is divisible by 5.
        Conversely, if we rearrange this,
        \[
            d_0 = n - 10d_1 - 10^2d_2 - \cdots - 10^kd_k,
        \]
        we see that if $n$ is divisible by 5, then the whole right-hand side (which is equal to $d_0$) is also divisible by 5.
    \end{enumerate}
\end{example}

We record some basic properties of divisibility here. The proof of this proposition is a straightforward exercise.

\begin{proposition}
    Let $a$, $b$ and $c$ be integers.
    \begin{enumerate}[(a)]
        \item If $a\mid b$ and $b\mid c$, then $a\mid c$.
        \item If $a\mid b$ and $b\mid c$, then $a = \pm b$.
        \item If $a\mid b$ and $a\mid c$, then $a\mid (b+c)$ and $a\mid (b-c)$.
    \end{enumerate}
\end{proposition}

\begin{question}
    For those familiar with equivalence relations, is divisibility an equivalence relation on $\Z$?
\end{question}

\begin{definition}
    A \emph{common divisor} of integers $a$ and $b$ is a positive integer $d$ that divides both of them.
    The \emph{greatest common divisor} of $a$ and $b$ is the largest positive integer $d$ such that $d\mid a$ and $d\mid b$ and we write $d = \gcd(a,b)$ or $d = (a,b)$ if there is no possibility of confusion.
\end{definition}

\begin{example}
    \begin{enumerate}[(a)]
        \item Find the greatest common divisor of $132$ and $66$ by listing out all of their divisors.

        \item Find the greatest common divisor of $80$ and $5$. Other than the number being pretty small, why was this easy to do? Prove your idea.
    \end{enumerate}
\end{example}

Of course given integers $a$ and $b$, it's not always the case that $a\mid b$ or $b\mid a$.
In this case, we get a (unique) remainder.

\begin{proposition}
    For any positive integers $a$ and $b$, there exist unique integers $q$ and $r$ such that
    \begin{equation}\label{division}
        a = bq + r\qquad \text{with }0\leq r < b.
    \end{equation}
    Here we call $q$ the \emph{quotient} and $r$ the \emph{remainder} when $a$ is divided by $b$.
\end{proposition}
\begin{proof}
    Homework exercise.
\end{proof}

Division with remainder provides us with a way of finding the gcd of two integers.
To see this, rearrange (\ref{division}) to obtain
\[
    r = a - bq.
\]
If $d$ is a common divisor of $a$ and $b$, then it clearly divides the right-hand side of this equation, so it must divide $r$ as well.
A similar rearrangement (which?) shows that if $c$ is a common divisor of $b$ and $r$, then it must also divide $a$.
We then have that the common divisors of $a$ and $b$ are the common divisors of $b$ and $r$, so we must have that
\[
    \gcd(a,b) = \gcd(b,r).
\]
This is great because if we assume that $a > b$, then we've reduced the problem of finding $\gcd(a,b)$ to finding the gcd of two smaller numbers, $b$ and $r$.
We can then repeat this: divide $b$ by $r$ to obtain
\[
    b = q'r + r',\qquad \text{with }0\leq r' < r.
\]
By the same reasoning, we have that
\[
    \gcd(a,b) = \gcd(b,r) = \gcd(r,r').
\]
Since the remainders are positive numbers that get strictly smaller after each division, we must eventually reach a remainder of zero. The remainder right before this one is then the gcd of $a$ and $b$.
\begin{example}\label{gcd ex}
    Let's compute $\gcd(12345, 11111)$.
    Even without a calculator it's sometimes easy to eyeball how many times one number goes into another.
    \begin{align*}
        12345 &= 11111\cdot 1 + 1234\\
        11111 &= 1234\cdot 9 + 5\\
        1234 &= 5\cdot 246 + 4\\
        5 &= 4\cdot 1 + 1\\
        4 &= 1\cdot 4 + 0
    \end{align*}
    The second-to-last remainder we found was 1, so we conclude that $\gcd(12345, 11111) = 1$.
    Note that even though the numbers involved started out somewhat large (for by-hand computations), we were able to calculate the gcd in just a few steps.
\end{example}

This procedure for computing the gcd of two integers is called the \emph{Euclidean algorithm} after the ancient Greek mathematician.
We summarize it here.
\begin{theorem}
    Let $a\geq b$ be positive integers.
    Then the following algorithm computes $\gcd(a,b)$ in a finite number of steps (i.e., the algorithm eventually terminates).

    \begin{algorithmic}[1]
        \State Let $r_0 = a$ and $r_1 = b$.
        \State Set $i = 1$. 
        \State Divide $r_{i-1}$ by $r_i$ with remainder to obtain quotient $q_i$ and remainder $r_{i+1}$.
        \[
            r_{i-1} = r_i\cdot q_i + r_{i+1},\qquad \text{with }0\leq r_{i+1}<r_i.
        \]
        \State If $r_{i+1} = 0$, then $r_i = \gcd(a,b)$ and the algorithm terminates.
        \State Otherwise, $r_{i+1} > 0$. Set $i = i+1$ and go to Step 3.
    \end{algorithmic}

\end{theorem}

How many times do we need to repeat the division step of the algorithm?
Let's start by looking at how much the remainders drop at each step.
At each step we have two possibilities: either $r_{i+1} \leq \frac{1}{2}r_i$ or $r_{i+1} > \frac{1}{2}r_i$.
In the first case, since the remainders are strictly decreasing, we have
\[
    r_{i+2} < r_{i+1} \leq \frac{1}{2}r_i.
\]
In the other case we must have $r_i = r_{i+1}\cdot 1 + r_{i+2}$. Rearranging, we have
\[
    r_{i+2} = r_i - r_{i+1} < r_i - \frac{1}{2}r_i = \frac{1}{2}r_i.
\]
In either case, we have that the remainder drops by at least half every two steps. After $2k+1$ steps we then have
\[
    r_{2k+1} < \frac{1}{2}r_{2k-1} < \frac{1}{2^2}r_{2k-3} < \cdots < \frac{1}{2^k}r_1 = \frac{1}{2^k}b.
\]
If $k$ is the smallest integer such that $b/2^k<1$, then we have $r_{2k+1} = 0$.
Setting $k = \lfloor \log_2 b\rfloor + 1$ does the trick.
The $gcd$ is then found on step at most $2k = 2\lfloor \log_2b\rfloor + 2$.

\begin{remark}
    Pretty much all cryptography software includes some implementation of the Euclidean algorithm.
    Computers store integers in their binary representations where an integer $N$ takes $n = \lfloor \log_2 N\rfloor +1$ bits of memory (why?). The above analysis shows that the Euclidean algorithm runs in a number of steps equal to at most twice the number of bits $(2n)$ it takes to store the smaller of its two inputs.
    When the number of steps it takes an algorithm to complete grows (at most) like a polynomial in its input size, then we consider it to be (reasonably) efficient.
\end{remark}

The Euclidean algorithm also gives us a way of writing $\gcd(a,b)$ as a linear combination of $a$ and $b$.

\begin{example}
    Let's return to Example \ref{gcd ex}.
    % We know $\gcd(12345, 11111) = 1$ since 1 is the second-to-last remainder.
    % Isolate the gcd and use the previous two equations to get the remainders $4$ and $5$ in terms of earlier remainders.
    % \begin{align*}
    %     1 &= 5 - 4\cdot 1\\
    %      &= (11111-1234\cdot 9) - (1234-5\cdot 246)\\
    %      &= 11111 -1234\cdot 10 + 5\cdot 246
    % \end{align*}
    % Now get the remainder 1234 in terms of the previous remainders and do the same for 5 once more.

    % \begin{align*}
    %     1 &= 11111 -1234\cdot 10 + 5\cdot 246\\
    %       &= 11111 - (12345 - 11111)\cdot 10 + (11111-1234\cdot 9)\cdot 246\\
    %       &= 12345\cdot (-10) + 11111\cdot 257 + 1234\cdot (2214)\\
    %       &= 
    % \end{align*}

    Write $a = 12345$ and $b = 11111$ and solve for the first remainder, 1234, in terms of $a$ and $b$:
    \[
        1234 = a - b.
    \]
    Now plug this into the second equation to get
    \[
        b = (a-b)\cdot 9 + 5,
    \]
    So the next remainder, 5, can be written in terms of $a$ and $b$ as
    \[
        5 = -9a + 10b.
    \]
    Plug this along with the expression for 1234 into the third equation to get
    \[
        a-b = (-9a + 10b)\cdot 246 + 4,
    \]
    which gives the next remainder, 4, in terms of $a$ and $b$:
    \[
        4 = 2215a - 2461b.
    \]
    Finally, plug the expressions for $4$ and $5$ into the second-to-last equation to get
    \[
        1 = (-9a + 10b) - (2215a - 2461b) = -2224a + 2471b.
    \]
\end{example}

This example is more or less a proof of the following theorem.

\begin{theorem}\label{bezout}
    Let $a$ and $b$ be positive integers.
    Then the equation
    \[
        ax + by = c
    \]
    has integer solutions for $x$ and $y$ if and only if $c$ is divisible by $\gcd(a,b)$.
    Moreover, if $(x_0, y_0)$ is a particular solution to this equation, then every other solution has the form
    \[
        x = x_0 + \frac{kb}{\gcd(a,b)},\quad y = y_0 - \frac{ka}{\gcd(a,b)}
    \]
    for some integer $k$.
\end{theorem}










\subsection{Modular Arithmetic}

Recall that when encrypting a message with a shift cipher with key $k$, each letter in the plaintext is shifted forward in the alphabet by $k$ positions.
Importantly, we \emph{wrap around} the alphabet if we shift past the letter \texttt{Z} (or whatever letter is at the end of the relevant alphabet).
This idea of wrapping around the end back to the beginning comes up in our day-to-day lives when we think about telling time.
Four hours after 9AM is 1PM since we \emph{wrap around} 12pm back to 1PM (the same idea holds if you prefer to think in military time - three hours after 2300 is 0200).
We'll look at this mathematically with the idea of \emph{congruence}.

\begin{definition}
    Let $m\geq 1$ be an integer. We say that the integers $a$ and $b$ are \emph{congruent modulo }$m$ if the difference $a-b$ is divisible by $m$.
    In this case we write
    \[
        a \equiv b \pmod{m}
    \]
    and call $m$ the \emph{modulus}.
\end{definition}


\begin{example}
    We have that $2 \equiv 5\pmod{7}$. We also have that $2\equiv 9\pmod 7$ and $2 \equiv 16\pmod 7$.
\end{example}

Importantly, congruences respect familiar operations like addition and multiplication, but are a little trickier when it comes to division.

\begin{proposition}\label{well defined}
    Let $m\geq 1$ be an integer.
    \begin{enumerate}
        \item If $a_1 \equiv a_2 \pmod m$ and $b_1 \equiv b_2\pmod m$, then
        \[
            a_1 \pm b_1 \equiv a_2 \pm b_2 \pmod m\qquad \text{and}\qquad a_1b_1\equiv a_2b_2\pmod m.
        \]

        \item Let $a$ be an integer. Then there exists an integer $b$ such that
        \[
            ab \equiv 1\pmod m
        \]
        if and only if $\gcd(a,m) = 1$.
        In this case, we call $b$ the \emph{multiplicative inverse of $a$ modulo $m$} and we write $b = a^{-1}\pmod m$.
    \end{enumerate}
\end{proposition}

\begin{proof}
    The proof of part (a) isn't super interesting, so we'll skip it.

    For part (b), first suppose that $\gcd(a,m) = 1$.
    Then by Theorem (\ref{bezout}), we can find $u$ and $v$ such that $au+mv = 1$.
    But if we rearrange this, we have
    \[
        au - 1 = mv,
    \]
    so the difference $au-1$ is divisible by $m$ and $au\equiv 1\pmod m$.
    In this case, $u$ is an inverse of $a\pmod m$.

    On the other hand, suppose $a$ has a multiplicative inverse $b\pmod m$.
    Then $m$ divides the difference $ab-1$ so we have
    \[
        ab - km = 1
    \]
    for some integer $k$.
    If $d$ is some (positive) common divisor of $a$ and $m$, then $d$ must divide the left-hand side of this equation.
    But then $d$ must divide 1, so we must have $d = 1$.
    It must then be the case that $\gcd(a, m) = 1$.
\end{proof}

Part (b) of this proposition gives us a partial analogue of division modulo $m$.
Just like how the rational number $1/2$ has the property that $(1/2)\cdot 2 = 1$, the number $3$ has the property that $3\cdot 2 = 6 \equiv 1\pmod 5$, so $3$ plays a similar role to $1/2$.
What's more is that our proof of part (b) gives us an algorithm for computing the modular inverse: the extended Euclidean algorithm.

\begin{example}
    Let's find the inverse of $4$ modulo 7 (if it exists at all).
    First compute $\gcd(4, 7)$.
    \begin{align*}
        7 &= 4\cdot 1 + 3\\
        4 &= 3\cdot 1 + 1\\
        3 &= 1\cdot 3 + 0
    \end{align*}
    So $\gcd(4, 7) = 1$, so we know a modular inverse exists.
    We find it by substituting in expressions for the remainders.

    \begin{align*}
        1 &= 4 - 3\cdot 1\\
          &= 4 - (7-4)\\
          &= 4\cdot 2 - 7.
    \end{align*}
    Rearranging this, we see that $4\cdot 2 - 1 = 7$, so $4\cdot 2\equiv 1\pmod 7$, and 2 is the inverse of 4 modulo 7.
\end{example}

Remember that the Euclidean algorithm is really efficient (for a computer at least - so is the extended one), so finding inverses is efficient as well.

Returning to the above example, note that $4\cdot 9 = 36 \equiv 1\pmod 7$ as well, so we can just as easily say that $9$ is an inverse of 4 modulo 7.
It would be nice if there was just one inverse or a way for two people to pick the same inverse every time.
Division with remainder gives us a way of doing this.
If $b$ is an inverse of $a$ modulo $m$, write
\[
    b = mq + r\quad\text{with }0\leq r < m.
\]
Then $r$ is always between 0 and $m-1$. Since this $r$ is unique, we can agree that we always work with the integers 0 through $m-1$ when we work modulo $m$.
This idea is encapsulated in the following proposition.

\begin{proposition}\label{remainders only}
    The integers $a$ and $b$ are congruent modulo $m$ if and only if they have the same remainder when divided by $m$.
\end{proposition}

Recall the notion of equivalence relations.

\begin{definition}
    A relation $\sim$ on a set $X$ is an \emph{equivalence relation} if the following all hold.
    \begin{enumerate}
        \item (Reflexivity) $x\sim x$ for all $x\in X$.
        \item (Symmetry) $x\sim y$ if and only if $y\sim x$ for any $x,y\in X$.
        \item (Transitivity) if $x\sim y$ and $y\sim z$ then $x\sim z$.
    \end{enumerate}
    For each $x\in X$, the \emph{equivalence class of }$x$, denoted $[x]$ (or sometimes $\overline{x}$) is
    \[
        [x] = \{y\in X: x\sim y\}.
    \]
    We can form the new set $X/\sim$, the \emph{quotient of $X$ by $\sim$} by just taking equivalence classes.
    \[
        X/\sim = \{[x]: x\in X\}.
    \]
\end{definition}

Modular arithmetic is a concrete example of this.

\begin{proposition}
    Fix a positive integer $m\geq 2$. Then equivalence modulo $m$ is an equivalence relation on $\Z$.
\end{proposition}

Moreover, Proposition (\ref{remainders only}) leads us to think that the quotient of $\Z$ by ``equivalence modulo $m$'' is the ``correct'' object to work with and to choose our equivalence classes to be $[0], \ldots, [m-1]$.

\begin{definition}
    The set $\Z/m\Z$ is defined to be the set of integers quotiented by the relation ``equivalent modulo $m$''. Specifically,
    \[
        \Z/m\Z = \{[0], \ldots, [m-1]\},
    \]
    where $[a] = \{b\in \Z: a\equiv b\pmod m\}$.
\end{definition}
\begin{remark}
    When working with $\Z/m\Z$, we usually drop the $[\cdot ]$ when talking about its elements, which are equivalence classes. That is, it technically doesn't make sense to write $2\in \Z/5\Z$ since 2 isn't an equivalence class. However, as the next proposition shows, the equivalence class $[2]$ can be made to behave a lot like the ordinary integer 2.
\end{remark}

We can carry the notions of addition and multiplication over to the quotient as well.

\begin{definition}
    For $[a], [b]\in \Z/m\Z$, define $[a] + [b]$ to be $[a+b]$ and $[a]\cdot [b]$ to be $[ab]$.
\end{definition}

\begin{remark}
    Technically, the above definition should be made into a proposition that says this definition is \emph{well-defined}.
    That is, we need to show that if $a\equiv a'$ and $b\equiv b'$ then we want $[a] + [b] = [a'] + [b']$ and $[a]\cdot [b] = [a']\cdot [b']$.
    This follows easily from Proposition (\ref{well defined}).
\end{remark}

Let's think about Theorem \ref{bezout} for a bit in this context by looking at equations in $\Z/m\Z$.

\begin{example}
    \begin{enumerate}
        \item Does $2x\equiv 3$ have a solution modulo 5?
        It would be nice if we could ``divide by 2'' and that's exactly what a multiplicative inverse lets us do.
        It's easy to verify that $4$ is the inverse of $2\pmod 5$, so multiplying both sides of this equation through by 4 gives $x \equiv 12 \equiv 2\pmod 5$.

        \item What about $2x\equiv 3\pmod 6$?
        This equation in $\Z/6\Z$ is equivalent to the integer equation $2x - 3 = 6y$,
        which has no solution since the left-hand side is always odd while the right-hand side is always even.
        Another way we can think about it is that we can't ``divide by 2'' since $\gcd(2, 6) = 2 \neq 1$.

        \item What about $2x \equiv 4 \pmod 6$?
        Just like in the last example, we can't divide by 2.
        However, it's easy to see that $x \equiv 2\pmod 6$ is a solution.
        But this solution isn't unique since $x \equiv 5 \pmod 6$ is also a solution.
    \end{enumerate}
\end{example}

In short, the existence of an inverse, as determined by Theorem \ref{bezout} determines whether or not equations like $ax \equiv b\pmod m$ have solutions.
If $\gcd(a, m) = 1$, then there's a unique solution.
Otherwise, there can either be no solution or there might be multiple solutions.
If we want to restrict ourselves to the (equivalence classes of) integers that \emph{do} have inverses modulo $m$, then we use the following object.

\begin{definition}
    Fix an integer $m\geq 2$. Then the set of \emph{units modulo $m$} is denoted by
    \begin{align*}
        (\Z/m\Z)^\times &= \{a\in \Z/m\Z: \gcd(a,m) = 1\}\\
        &= \{a\in \Z/m\Z: a\text{ has an inverse modulo }m\}.
    \end{align*}
\end{definition}










\subsection{Prime Numbers, Unique Factorization, and Finite Fields}

The ``building blocks'' of the integers are the prime numbers.
\begin{definition}
    An integer $p$ is called \emph{prime} if $p\geq 2$ and if the only positive integers dividing $p$ are 1 and $p$.
\end{definition}

Note that if $p$ is prime, then $\gcd(a,p) = 1$ for each $1\leq a < p$ (why?).
Consequently, each nonzero element of $\Z/p\Z$ has a multiplicative inverse, i.e.
\[
    (\Z/p\Z)^\times = \{1, 2, \ldots, p-1\}.
\]
The set $\Z/p\Z$ forms a structure that we call a (finite) \emph{field}: a set where we can add and subtract as well as multiply and divide by (nonzero) elements.
We denote this field by $\F_p$.
Other examples of fields include $\R$ and $\Q$ but not $\Z$.

\begin{proposition}\label{prime divides product}
    Let $p$ be a prime number and suppose that $p \mid ab$.
    Then $p\mid a$ or $p\mid b$.
    More generally, if
    \[
        p\mid a_1a_2\cdots a_k,
    \]
    then $p\mid a_i$ for some $i$.
\end{proposition}
\begin{proof}
    We'll prove the first statement and you'll prove the second one in discussion.
    If $p$ divides $a$ then we're done.
    If $p\nmid a$, then $\gcd(a,p) = 1$ (why?), so we can write
    \[
        au + pv = 1
    \]
    for some integers $u$ and $v$.
    Multiplying this through by $b$ gives
    \[
        abu + pbv = b.
    \]
    By assumption, $p\mid ab$ and clearly $p\mid pbv$, so $p$ divides the left-hand side of this equation. Consequently, $p$ divides the right-hand side, which is $b$.
\end{proof}

Using this, we can prove what we said earlier about primes being ``building blocks.''

\begin{theorem}[The Fundamental Theorem of Arithmetic]
    Let $a\geq 2$ be an integer. Then $a$ can be factored as a product of prime numbers
    \begin{equation}\label{factorization}
        a = p_1^{e_1}p_2^{e_2}\cdots p_r^{e_r}
    \end{equation}
    for some positive integer $r$.
    Furthermore, this factorization is unique up to rearrangement of the primes.
\end{theorem}
\begin{proof}
    We prove that we can factor into primes by induction and uniqueness will come later.
    Our base case is $a=2$, and this itself is a prime factorization since 2 is prime.
    Suppose then that every integer less than $a$ can be factored into primes.
    If $a$ itself is prime, then we're done by the same reasoning we used in the base case.
    Otherwise, $a = bc$ where $1 < b,c < a$.
    By the induction hypothesis, we can factor $b$ and $c$ into primes:
    \[
        b = p_1^{e_1}\cdots p_k^{e_k},\quad c= q_1^{f_1}\cdots q_\ell^{f_\ell}.
    \]
    But then $a = p_1^{e_1}\cdots p_k^{e_k}q_1^{f_1}\cdots q_\ell^{f_\ell}$ is a factorization of $a$.
    You'll prove the uniqueness part of this statement in discussion.
\end{proof}

Looking at the factorization of $a$ into primes (\ref{factorization}), we call the number of times a particular prime, $p$, appears in the factorization the \emph{order of $p$ in $a$} and denote it by $ord_p(a)$.
That is, in the factorization (\ref{factorization}), $ord_{p_i}(a) = e_i$.











\subsection{Powers and Primitive Roots in Finite Fields}


We can add, subtract, multiply and (sometimes) divide by elements of $\Z/m\Z$.
Since we can multiply, we can definitely exponentiate in exactly the way you think we would.
If $a\in \Z/m\Z$ and $k$ is a nonnegative integer, then $a^k$ is the product of $a$ with itself $k$ times, taken modulo $k$.
Moreover, if $a$ has inverse $a^{-1}$, then we can define negative powers of $a$ as positive powers of $a^{-1}$.
We need to be a little careful though.
We can raise an element of $\Z/m\Z$ to an integer power, but it doesn't make sense to raise an element of $\Z/m\Z$ to the power of another element of $\Z/m\Z$.

\begin{example}
    We clearly have that $2^1 \equiv 2\pmod 5$.
    However, $2^5 = 32\equiv 2\pmod 5$ even though $5 \not\equiv 1\pmod 5$.
    In other words, $a^b \equiv a^c \pmod m$ \emph{does not} imply that $b \equiv c\pmod m$.
\end{example}

A few of the main cryptographic protocols we'll talk about come from the properties of modular exponentiation, so let's talk a bit about it.
Let's look at some of the powers of 2 modulo 7
\[
    2^1 \equiv 2,\ 2^2 \equiv 4,\ 2^3 = 8 \equiv 1,\ 2^4 = 16 \equiv 2,\ 2^5 = 32 \equiv 4,\ 2^6 = 64 \equiv 1, \ldots
\]
It looks like we get a repeating pattern of 1, 2, 4.
In fact, we can prove that this pattern holds true: take any positive integer $k$ and divide it by 3 with remainder to get $k = 3q + r$. Then
\[
    2^k = 2^{3q + r} = (2^3)^q\cdot 2^r \equiv 1^q\cdot 2^r \equiv 2^r\pmod 7.
\]
That is, the value of $2^k$ only depends on the remainder we get when we divide $k$ by 3, i.e. we care about what $k$ is modulo 3, \emph{not} modulo 7.
What happens with the powers of other numbers, say 3 (still modulo 7)?
\[
    3^1 = 3,\ 3^2 \equiv 2,\ 3^3\equiv 6,\ 3^4 \equiv 4,\ 3^5\equiv 6,\ 3^6\equiv 1,\ 3^7\equiv 3,\ldots
\]
Like with 2, we have that $3^6 \equiv 1\pmod 7$.
However, it looks like we get a repeating pattern of length six this time.
What's more is that the powers of 3 give us all the nonzero elements of $\Z/7\Z$.

Let's solidify the first of these observations into a theorem.
\begin{theorem}[Fermat's Little Theorem]
    Let $p$ be a prime number and let $a$ be any integer. Then
    \[
        a^{p-1} \equiv \begin{cases}
            1\pmod p,&\text{if }p\nmid a,\\
            0\pmod p,&\text{if }p\mid a.
        \end{cases}
    \]
\end{theorem}
\begin{proof}
    If $p$ divides $a$ then it divides every power of $a$, so let's just look at the case where $p\nmid a$.
    Let's look at the numbers
    \begin{equation}\label{distinct}
        a,\ 2a,\ 3a,\ \ldots,\ (p-1)a,
    \end{equation}
    We claim that these are all \emph{distinct} when reduced modulo $p$.
    Indeed, if $ka \equiv ja\pmod p$, then $p$ divides $a(k-j)$.
    By Proposition \ref{prime divides product}, $p$ must then divide $a$ or $k-j$.
    Since we've assumed that $p\nmid a$, we must have that $p$ divides $k-j$.
    But we haven't listed any multiples of $p$ above, so we must have $k=j$.

    Now let's multiply all the elements in (\ref{distinct}) together.
    On one hand, this is clearly $a^{p-1}\cdot (p-1)!$.
    On the other hand, since since these are $p-1$ distinct nonzero integers between 1 and $p-1$, the must be all of the integers in this range, so their product is $(p-1)!$.
    We must then have
    \[
        a^{p-1}\cdot (p-1)! \equiv (p-1)!\pmod p.
    \]

    We can then cancel the $(p-1)!$ from both sides (why?) to obtain $a^{p-1}\equiv 1\pmod p$.
\end{proof}

This theorem has some really powerful implications for computation.

\begin{example}
    The integer $p = 15485863$ is prime, so by Fermat's little theorem we have
    \[
        2^{15485862} \equiv 1\pmod {15485863}.
    \]
    Even though the numbers involved are large ($2^{15485862}$ has more than 400,000 digits), we can write the above identity without doing any real computation (we had to know that 15485863 is prime first, and we'll see some good algorithms for verifying this later).
\end{example}

Fermat's little theorem tells us that if $p\nmid a$, then $a^{p-1}\equiv 1\pmod p$, but as we saw with the powers of 2 modulo 7, a smaller power of $a$ might be congruent to 1 modulo $p$.
This motivates the following definition.

\begin{definition}
    Let $p\geq 2$ be prime. For any integer $a$ such that $p\nmid a$, the \emph{order of $a$ modulo $p$} is the smallest positive integer $k$ such that $a^k\equiv 1\pmod p$.
\end{definition}

Fermat's little theorem tells us that that the order of $a$ is at most $p-1$ so long as $p\nmid a$.
The following proposition claims that the order of $a$ modulo $p$ can't be just anything however.

\begin{proposition}
    Let $p$ be a prime and let $a$ be an integer with $p\nmid a$.
    If $a^n\equiv 1\pmod p$, then the order of $a$ modulo $p$ divides $n$.
    In particular, the order of $a$ divides $p-1$.
\end{proposition}

\begin{proof}
    Suppose $k$ is the order of $a$ modulo $p$.
    We divide $n$ by $k$ to obtain
    \[
        n = kq + r
    \]
    for some $0\leq r < k$.
    We then have
    \[
        1 \equiv a^n \equiv a^{kq + r} \equiv (a^k)^q\cdot a^r \equiv 1^q\cdot a^r \equiv a^r\pmod p.
    \]
    But $k$ is the smallest positive power of $a$ congruent to 1 modulo $p$, so we must have $r = 0$ and $k\mid n$.
\end{proof}

Looking at the powers of 3 modulo 7, we see that sometimes the powers of $a$ modulo $p$ can fill out all of the nonzero residues modulo $p$.
The following theorem says that there is always such a $p$ and you'll prove it on your next homework assignment.
\begin{theorem}[Primitive Root Theorem]
    Let $p$ be prime.
    Then there exists an element $g \in \F_p^\times$ such that
    \[
        \F_p^\times = \{1, g, g^2, \ldots, g^{p-2}\}.
    \]
    Such a $g$ is called a \emph{generator} or \emph{primitive root of }$\F_p$.
\end{theorem}


Let's talk a little about how to actually compute $a^k\pmod m$.
The naive way, repeated multiplication, would simply compute $a^i$ for all $i\leq k$:
\[
    a_1 \equiv a\pmod m,\quad a_2 \equiv a\cdot a_1 \pmod m,\quad a_3 \equiv a\cdot a_2\pmod m,\quad \ldots\quad a_k = a\cdot a_{k-1}\pmod m.
\]
If $k$ is of moderate size (for a computer), say around 1000 bits (around 300 digits), then the time it would take to complete this algorithm would be greater than the estimated age of the universe, even if you reduced modulo $m$ after each step (if you didn't, then the integer $a^k$ would take up more bits than there are particles in the universe by some estimates).
Let's look at an example for how to compute large powers very efficiently.

\begin{example}
    Let's compute $3^{75}\pmod {100}$.
    We start by writing $75$ in binary.
    The largest power of 2 that is no larger than 75 is 64, so
    \[
        75 = 64 + 11.
    \]
    Now the largest power of 2 at most 11 is 8. We repeat this process.
    \begin{align*}
        76 &= 64 + 8 + 3\\
        &= 64 + 8 + 2 + 1.
    \end{align*}
    Using this we can write
    \begin{equation}\label{3 to binary}
        3^{76} = 3^{64 + 8 + 2 + 1} = 3^{64}\cdot 3^8\cdot 3^2\cdot 3^1.
    \end{equation}
    Now we can compute the seven numbers
    \[
        3,\ 3^2,\ 3^4,\ 3^8,\ \ldots,\ 3^{64}
    \]
    modulo 100 quite easily - each number is just the square of the one before it.
    Now using (\ref{3 to binary}), we decide which of these powers of 3 to multiply together.
    \begin{center}
    \begin{tabular}{|c|| c | c | c | c | c | c | c | c|}
        \hline
        $i$ & 0 & 1 & 2 & 3 & 4 & 5 & 6\\
        \hline
        $3^{2^i}\pmod{100}$ & 3 & 9 &  81 & 61 & 21 & 41 & 81\\
        \hline
    \end{tabular}
    \end{center}
    This gives
    \begin{align*}
         3^{76} &= 3^1\cdot 3^2\cdot 3^8\cdot 3^{64}\\
         &\equiv 3\cdot 9\cdot 61\cdot 81 \pmod{100}\\
         &\equiv 21 \pmod{100}.
    \end{align*}
\end{example}


Let's describe this algorithm, sometimes called the \emph{fast powering algorithm} or the \emph{square-and-multiply algorithm}, more formally.

\begin{enumerate}
\item \textbf{Given: }integers $g$, $A$ and $N$
\item Compute the binary expansion of $A$ as 
\[
    A = A_0 + A_1\cdot 2 + A_2\cdot 2^2 + \cdots + A_r\cdot 2^r,\quad \text{with }A_i\in \{0,1\}\text{ for all }i.
\]
\item Compute the powers $A^{2^i}\pmod m$ for each $0\leq i \leq r$ by squaring.
    \begin{align*}
        a_0 &\equiv g\pmod N\\
        a_1 &\equiv a_0^2\equiv g^2\pmod N\\
        a_2 &\equiv a_1^2\equiv g^{2^2}\pmod N\\
        \vdots\\
        a_r &\equiv a_{r-1}^2\equiv g^{2^r}\pmod N.
    \end{align*}

\item Compute $g^A\mod N$ by multiplication.
    \begin{align*}
        g^A &= g^{A_0 + A_1\cdot 2 + A_2 \cdot 2^2 + \cdots + A_r\cdot 2^r}\\
        &= g^{A_0}\cdot (g^2)^{A_1}\cdot (g^{2^2})^{A_2}\cdots (g^{2^r})^{A_r}\\
        &\equiv a_0^{A_0}\cdot a_1^{A_1} \cdots a_r^{A_r}\pmod n.
    \end{align*}
\end{enumerate}








\addtocounter{subsection}{1}
\subsection{Symmetric and Asymmetric Ciphers}
Let's briefly formalize some notions from cryptography.
Suppose Alice and Bob agree to send and receive messages using some cipher.
If the same key is used for encryption and decryption, then we say that the cipher is \emph{symmetric}.
We can describe a symmetric cipher with these sets and maps.
\begin{align*}
    \mathcal{K} &= \text{all possible keys}\\
    \mathcal{M} &= \text{all possible plaintexts}\\
    \mathcal{C} &= \text{all possible ciphertexts}\\
    e &: \mathcal{K}\times \mathcal{M}\to \mathcal{C}\\
    d &: \mathcal{K}\times \mathcal{C}\to \mathcal{M}.
\end{align*}
The function $e$ is the \emph{encryption function} that takes in a key, plaintext pair $(k, m)$ and outputs the corresponding ciphertext.
Likewise, the \emph{decryption function} $d$ takes a key, ciphertext pair $(k, c)$ and outputs the plaintext.

Of course we want decryption to undo encryption with the same key, so we require the following consistency condition.
\[
    d(k, e(k, m)) = m,\qquad \text{for all }k\in \mathcal{K},\ m\in \mathcal{M}.
\]
Another way to think about this condition is to define encryption and decryption functions $e_k$, $d_k$ for each key $k$,
\[
    e_k(\cdot) = e(k, \cdot),\qquad d_k(\cdot) = d(k, \cdot)
\]
Then the consistency condition just says that $d_k$ is the inverse function of $e_k$.
This of course requires that $e_k$ be injective.
\begin{example}
    The substitution cipher is a symmetric cipher.
\end{example}

When assessing the security of a cipher, we \emph{always assume that an eavesdropper, Eve, knows the method of encryption.}
That is, we assume Eve knows $e$ and $d$, but not the key $k$.
Under this assumption, here are some of the properties we want for $(\mathcal{K}, \mathcal{M}, \mathcal{C}, e, d)$.
\begin{enumerate}
    \item Easy encryption: for any $k\in \mathcal K$ and $m\in \mathcal M$, it's easy to compute $e_k(m)$.
    \item Easy decryption: for any $k\in \mathcal K$ and $c \in \mathcal C$, it's easy to compute $d_k(c)$.
    \item Strength against ciphertext-only attacks: If Eve knows that the ciphertexts $c_1, c_2, \ldots, c_n\in \mathcal C$ were encrypted using the same (unknown) key $k$, it should be hard for her to compute any of the corresponding plaintexts, $d_k(c_1), \ldots, d_k(c_n)$ without knowing $k$.

    \item Strength against known-plaintext attacks: If Eve knows some plaintexts and their corresponding ciphertexts $(m_1, c_1), \ldots, (m_n, c_n)$, where $c_i = e_k(m_i)$, then it should be hard for her to compute $d_k(c)$ for any new ciphertext $c$ without knowing $k$.

    \item Strength against chosen-plaintext attacks: If Eve \emph{chooses} some plaintexts $m_1, \ldots, m_n$ and knows their corresponding ciphertexts $c_1, \ldots, c_n$, it should still be hard for her to compute $d_k(c)$ for any new ciphertext $c$ without knowing $k$.

    \item Strength against chosen-ciphertext attacks: If Eve chooses some ciphertexts $c_1, \ldots, c_n$ and knows their corresponding plaintexts $m_1, \ldots, m_n$, it should still be hard for her to compute $d_k(c)$ for any new ciphertext $c$ without knowing $k$.
\end{enumerate}

\begin{example}
    The substitution cipher isn't very resistant to ciphertext-only attacks since it's vulnerable to frequency analysis when we have a long ciphertext.
    It's also quite vulnerable to known-plaintext attacks since we can just query a few messages to learn most of the substitutions.
\end{example}

\begin{remark}
    When we think of plaintexts we often think of strings of Latin characters.
    However, we can usually \emph{encode} written characters into numbers (indeed, this is what computers do), so we can safely think of $\mathcal M$ and $\mathcal C$ as sets of numbers or elements of $\Z/m\Z$.
\end{remark}

\begin{example}
    Let $\mathcal M = \mathcal C = \mathcal K = \F_p^\times$.
    For any key $k$, define encryption by
    \[
        e_k(m) = m\cdot k\pmod p.
    \]
    We can assume that basic arithmetic like addition and multiplication modulo $p$ is efficient to compute.
    We can also efficiently decrypt, since the Euclidean algorithm gives us a way to compute $k^{-1}\pmod p$ in around $2\log_2 p$ steps, so
    \[
        d_k(c) = k^{-1}\cdot c \pmod p
    \]
    is efficient to compute.
    This cipher is \emph{completely broken} (that is, we can learn the key) by a known plaintext attack.
    If we know $c$ and $m$, then we can simply compute
    \[
        k \equiv c^{-1}\cdot m\pmod p
    \]
    since we can efficiently compute $c^{-1}\pmod p$.
\end{example}

One issue with symmetric ciphers is how Alice and Bob agree on a key.
If they have to exchange the key in the open, then it's vulnerable to Eve.
Alternatively, they can meet in person, but this can be inconvenient or impossible.
This motivates the idea of \emph{public-key cryptography}.
Here, we have two separate keys, $k = (k_{priv}, k_{pub})$.
The \emph{public key}, $k_{pub}$ is known to everyone, while the \emph{private key} is known only to the person receiving messages.
We have encryption and decryption functions
\begin{align*}
    e_{k_{pub}} &: \mathcal M\to \mathcal C\\
    d_{k_{priv}} &: \mathcal C\to \mathcal M,
\end{align*}
that are inverse to one another.
It should be difficult to compute $d_{k_{priv}}$ without knowledge of $k_{priv}$ even if one knows $k_{pub}$.

If Bob wants to send Alice a message, then Alice first sends Bob her public key, $k_{pub}$ (or maybe she publishes it in some public place like some website).
Note that we should then assume that Eve knows $k_{pub}$ as well.
Bob then computes $e_{k_{pub}}(m)$ and sends it to Alice.
Alice then decrypts this using $k_{priv}$, which only she knows.
The private key $k_{priv}$ is sometimes called \emph{trapdoor information} for the \emph{one-way} function $e_{k_{pub}}$ because it should be hard to invert this function without knowledge of $k_{priv}$.









\section{Discrete Logarithms and Diffie-Hellman}
\addtocounter{subsection}{1}
\subsection{The Discrete Logarithm Problem}
Public-key cryptography revolves around functions that are easy to compute but hard to invert without some special trapdoor information.
We've seen that exponentiation modulo $p$ is easy to compute by the square-and-multiply algorithm and invertible by the primitive root theorem.
It turns out that it's also hard to invert.

\begin{definition}
    Let $g$ be a primitive root of $\F_p^\times$ and let $h\neq 0$ in $\F_p$.
    The \emph{discrete logarithm problem (DLP)} is the problem of finding an exponent $x$ such that
    \[
        g^x\equiv h\pmod p.
    \]
    The number $x$ is called the \emph{discrete logarithm of $h$ to the base $g$} and is defined modulo $p-1$ by Fermat's little theorem.
\end{definition}

Remember that since $g$ is a primitive root we have that
\[
    \F_p^\times = \{g^0, g^1, \ldots, g^{p-2}\},
\]
so $h \equiv g^i$ for some $0 \leq i \leq p-2$.
By Fermat's little theorem we have that $g^{p-1}\equiv 1\pmod p$, so
\[
    g^{i + k(p-1)} \equiv h\pmod p
\]
for any integer $k$.
This\footnote{If you've taken complex analysis, this might remind you of the fact that $e^{r + 2\pi i} = e^r$ in $\C^\times$ and why we have to take so-called \emph{branch cuts} of the logarithm in that setting.} is why we say that the discrete logarithm is defined modulo $p-1$.
In particular, there are infinitely many integers $x$ such that $g^x\equiv h\pmod p$ and the DLP asks us to find just one such $x$.
This motivates us to define the function
\[
    \log_g: \F_p^\times \to \Z/(p-1)\Z
\]
for any primitive root $g$.

It makes sense to call this function a logarithm because it behaves how you would hope a logarithm does.
\begin{proposition}
    For any primitive root $g$ of $\F_p^\times$ and any $a,b\in \F_p^\times$ we have that
    \[
        \log_g(ab) \equiv \log_g(a) + \log_g(b)\pmod{p-1}.
    \]
\end{proposition}
\begin{proof}
    You'll prove this in your discussion section.
\end{proof}

How do we compute discrete logarithms?
Let's look at the brute-force solution.
For any $A\in \Z/(p-1)\Z$ we can compute $g^A\pmod p$ in around $2\log_2p$ multiplications (steps), and if try to solve the DLP by trial and error, we'd compute
\[
    g^0,\ g^1,\ g^2, \ldots
\]
modulo $p$ until we find $h$.
If $h \equiv g^{p-2}$ then this will take around $2p\log_2p$ steps.
If $p$ is even moderately large, say at least $2^{100}$, then this will take way too many steps for even modern computers.

\begin{remark}
    In some ways exponentiation and logarithms have different properties when we're working in $\F_p$ instead of $\R$ or $\C$.
    Even though they behave the same way \emph{algebraically} (i.e. they follow the same ``rules''), exponentiation modulo $p$ appears to behaves ``randomly'' in its input (of course it isn't actually random, but if you were to graph the function $x\mapsto g^x \pmod p$ for a primitive root $g$, then you'll get a random-looking cloud of points).
\end{remark}

\begin{remark}
    We can talk about the DLP even if $g$ isn't a primitive root modulo $p$.
    If it isn't, then the DLP asks us to find $x$ such that $g^x\equiv h\pmod p$ if such an $x$ exists.

    In fact, we can talk about the DLP in the setting of an arbitrary group.
    If $G$ is a group (written multiplicatively), then the DLP for $G$ asks us to find, for any two given $g$ and $h$ in $G$, an integer satisfying
    \[
        \underbrace{g\cdot g \cdot \cdots \cdot g}_{x\text{ times}} = h.
    \]
\end{remark}













\subsection{Diffie-Hellman Key Exchange}
Recall that with a symmetric cipher, Alice and Bob need to both have a key $k$ that they use to encrypt or decrypt messages.
Even if the cipher that they use is extremely resistant to any attack you can think of, if they exchange the key in an insecure way, then their communications are as good as compromised.
Whitfield Diffie and Martin Hellman released a paper in 1976 showing a way of doing this with modular exponentiation, but it's since been claimed that the British GCHQ knew of this idea since the sixties and kept it secret.

The following protocol allows for Alice and Bob to establish a shared secret even in the presence of an adversary, Eve.
The idea is that they can then use this secret as a key (or somehow use it to make a key) for a symmetric cipher.

\begin{enumerate}
    \item (Public) Alice and Bob both publicly agree on a large (around $2^{1000}$) prime $p$ and an integer $g$ relatively prime to $p$.
    Usually, $g$ is a primitive root modulo $p$, but this isn't necessary.

    \item (Private) Alice chooses a secret integer $a$ and computes $A \equiv g^a\pmod p$.
    Likewise, Bob chooses a secret integer $b$ and computes $B \equiv g^b\pmod p$.
    They perform this part completely independently of each other.
    Alice and Bob can complete this step efficiently with the square-and-multiply algorithm.

    \item (Public) Alice sends $A$ to Bob and Bob sends $B$ to Alice.
    This happens publicly (sometimes said ``in the clear''), so Eve can see this.

    \item (Private) Alice uses her secret integer to compute $B^a \pmod p$ and Bob likewise computes $A^b\pmod p$.
    Again, this can be done efficiently with square-and-multiply.
\end{enumerate}

After these steps are completed, Alice and Bob share the following value
\begin{align*}
    A^b &\equiv (g^a)^b \pmod p\\
    &\equiv (g^b)^a\pmod p\\
    &\equiv B^a\pmod p.
\end{align*}

Eve has access to anything that was shared publicly, namely $p$, $g$, $A$, and $B$.
If Eve knew either $a$ or $b$, then she could simply compute $B^a$ or $A^b$ modulo $p$ to compromise the shared secret.
But in order for her to do this, she needs to solve either of the equations
\[
    g^x \equiv A\pmod p,\qquad g^y\equiv B\pmod p.
\]
That is, if Eve can solve the discrete logarithm problem, then she can find the shared secret.

\begin{example}
    Suppose Alice and Bob agree on the prime $p=13$ and the integer $g = 2$.
    Next, say Alice randomly chooses $a=7$ and Bob randomly (and independently) chooses $b = 9$.
    Alice computes $A\equiv g^a = 2^7 \equiv 11\pmod{13}$ and Bob computes $B\equiv g^b = 2^9 \equiv 5\pmod {13}$.
    Alice and Bob then exchange $A$ and $B$ in the clear and compute
    \[
        B^a = 5^7\equiv 8\pmod {13},\qquad A^b = 11^9\equiv 8\pmod {13}.
    \]
    If Eve can find $x$ such that $2^x\equiv 11\pmod {13}$ or $y$ such that $2^y\equiv 5\pmod{13}$, then she too can perform one of the above computations to arrive at the shared value of 8.
\end{example}

If Eve can solve the DLP, then she can compromise Diffie-Hellman key exchange.
Does she really have to do this though?
Really, what she needs to do is solve the following problem.

\begin{definition}
    Let $p$ be prime and let $g\in \F_p^\times$, if for any $a$ and $b$ we can use $g^a\pmod p$ and $g^b\pmod p$ to compute $g^{ab}\pmod p$, then we can solve the \emph{Diffie-Hellman (search) problem} or \emph{DHP}.
\end{definition}

\begin{remark}
    The Diffie-Hellman problem doesn't actually require us to find either of the exponents $a$, $b$.
\end{remark}
Clearly DHP is no harder than DLP, but it's \emph{unknown} whether or not solving the DHP would allow one to solve the DLP as well.








\subsection{The Elgamal Public Key Cryptosystem}
The Diffie-Hellman protocol serves as a way for Alice and Bob to create a key to use in some other symmetric cipher.
That is, it doesn't directly allow Alice and Bob to send messages to one another.
The following cryptosystem, attributed to Egyptian cryptographer Taher Elgamal (1985), while not the first of its kind (that was RSA in 1978 - we'll talk about that later), has a similar flavor to Diffie-Hellman key exchange.
The Elgamal cryptosystem proceeds as follows.

\begin{enumerate}
    \item (Public) Alice or some trusted party publicly chooses a large prime $p$ and an element $g \in \F_p^\times$ with large prime order. (In particular, $g$ is \emph{not} a primitive root. Why not?)

    \item (Key generation) Alice chooses a random $1\leq a\leq p-1$ and computes $A \equiv g^a\pmod p$, her \emph{public key}.
    Alice then publishes $A$ while keeping $a$, her \emph{private key} secret.

    \item (Encryption) Bob chooses a plaintext $m\in \F_p^\times$.
    He then chooses a random $1\leq k\leq p-1$, his \emph{ephemeral (temporary) key}, and computes $c_1 \equiv g^k\pmod p$ and $c_2 \equiv m\cdot A^k\pmod p$.
    He then sends the ciphertext $(c_1, c_2)$ to Alice.

    \item (Decryption) Alice uses her private key $a$ to compute the plaintext in the following way.
    \[
        (c_1^a)^{-1}\cdot c_2\equiv (g^{ak})^{-1}\cdot m\cdot A^k \equiv g^{-ak}\cdot g^{ak}\cdot m\equiv m\pmod p.
    \]
\end{enumerate}

Step 1 is interesting in its own right, but we'll talk about it more later when we discuss primality testing.
If we assume that Alice can efficiently generate random numbers (which can be done, but we won't talk about it here), then she can efficiently compute $A\equiv g^a\pmod p$ with the square-and-multiply algorithm, so step 2 is efficiently doable.
Similarly, Bob can efficiently compute $c_1 \equiv g^k\pmod p$ and $c_2 \equiv m\cdot A^k\pmod p$ in step 3.
Finally, Alice can use square-and-multiply to efficiently compute $c_1^a$ and then use, say, the Euclidean algorithm to compute its inverse, so decryption is also efficient.

What does an eavesdropper, Eve, need to do in order to compromise Elgamal?
If she can recover the ephemeral key $k$ that Bob uses to encrypt the plaintext $m$, then she can recover the plaintext by computing 
\[
    c_2\cdot (A^k)^{-1} \equiv m\cdot A^k\cdot A^{-k} \equiv m\pmod p.
\]
If she knows $k$, then she can do this efficiently.
If Eve can solve the discrete logarithm problem, then she can find $k$ given the public quantities $p$ and $c_1 \equiv g^k\pmod p$.
Similarly, if she can recover Alice's private key $a$, then she can decrypt any messages sent to Alice (in a way, she becomes indistinguishable from Alice).
Just like with $k$, if Eve can solve the discrete logarithm problem, then she can recover $a$ from $A \equiv g^a\pmod p$.
This proves the following proposition.

\begin{proposition}
    If Eve can solve the discrete logarithm problem then she can decrypt arbitrary Elgamal ciphertexts encrypted using arbitrary Elgamal public keys.
\end{proposition}

Computational problems like this are often stated in terms of \emph{oracles}.
We won't define this term rigorously, but you can think of an oracle for a computational problem as a machine or entity that, when given an instance of that problem, solves it instantly.
So another way of stating the above proposition is that Eve can decrypt arbitrary Elgamal ciphertexts with an oracle for the discrete logarithm problem.

How does the security of Elgamal compare to that of Diffie-Hellman.
Both of them can be broken by access to a discrete logarithm oracle, but can we use one to break the other?

\begin{proposition}
    Fix a prime $p$ and an element $g\in F_p^\times$ to use in Elgamal encryption.
    Suppose that Eve has access to an oracle that decrypts arbitrary Elgamal ciphertexts encrypted using arbitrary Elgamal public keys.
    Then she can use the oracle to solve the Diffie-Hellman problem.

    Conversely, if Eve has access to an oracle that solves the Diffie-Hellman problem, then she can decrypt arbitrary Elgamal ciphertexts.
\end{proposition}

\begin{proof}
    We'll show how to use an Elgamal oracle to break Diffie-Hellman and you'll do the converse on your homework.
    Recall that in the Diffie-Hellman problem, Eve is given
    \[
        A\equiv g^{a}\pmod p\qquad\text{and}\qquad B\equiv g^b\pmod p
    \]
    and she needs to compute $g^{ab}\pmod p$.
    The tool Eve has at her disposal is an Elgamal oracle, which takes as input a prime $p$, a base $g$, a public key $A$ and a ciphertext $(c_1, c_2)$ and returns the quantity
    \[
        (c_1^a)^{-1}\cdot c_2\pmod p.
    \]
    What inputs can Eve feed to her Elgamal oracle to solve the DHP?
    If she inputs the public key $A\equiv g^a\pmod p$ and the ciphertext $(B, 1)$, then the oracle will return
    \[
        (B^a)^{-1}\cdot 1 \equiv g^{-ab}\pmod p.
    \]
    Then she just has to invert this (say, with the Euclidean algorithm).
    Even if the oracle rejects inputs where the second part of the ciphertext is 1, she can still win.
    Eve can choose a random element $c_2\in \F_p^\times$ and submit $(B, c_2)$ to the oracle.
    The oracle will then return
    \[
        (B^a)^{-1}\cdot c_2 \equiv g^{-ab}\cdot c_2\pmod p.
    \]
    Then Eve just has to multiply this by $c_2^{-1}$ and invert to obtain $g^{ab}$.
\end{proof}











\subsection{Basic Group Theory}
Recall that we can add and multiply elements of $\F_p$.
If we just look at the additive structure, then we have an identity ($0 + a = a$ for all $a$) and (additive) inverses (for all $a$, $a + (-a) = 0$).
Similarly, if we look at $\F_p^\times$, we have an identity ($a\cdot 1 = a$ for all $a$) and (multiplicative) inverses (for all $a$, $a\cdot a^{-1} = 1$).
These kinds of structures that come with an operation that admits an identity and inverses come up a lot in math and merit having their own definition.

\begin{definition}
    A \emph{group} consists of a set $G$ along with a binary operation $\star: G\times G\to G$ that satisfies the following properties.
    \begin{enumerate}
        \item (Identity) There is an element $e\in G$ such that
        \[
            e \star a = a\star e = a
        \]
        for all $a\in G$.

        \item (Inverses) For each $a\in A$ there is a (unique) element $a^{-1}\in G$ such that
        \[
            a\star a^{-1} = a^{-1}\star a = e.
        \]

        \item (Associativity) For all $a$, $b$, $c$ in $G$ we have
        \[
            a\star (b\star c) = (a\star b)\star c.
        \]
    \end{enumerate}
\end{definition}

\begin{remark}
    We often drop the $\star$ from our notation and just write elements of $G$ next to each other to indicate the operation (called \emph {juxtaposition}).
    That is, we often write $ab$ instead of $a\star b$ (we write the group \emph{multiplicatively}).
    In this case, we sometimes write $1$ for the identity element.
    If the group is \emph{commutative}, i.e. $a\star b = b\star a$ for all $a$ and $b$, then we sometimes write $a+b$ instead of $a\star b$ (we write the group \emph{additively}) and 0 instead of $e$ for the identity.
\end{remark}

\begin{remark}
    If we write our group multiplicatively, then we write
    \[
        a^n = \underbrace{a\cdot a\cdot \cdots \cdot a}_{n\text{ times}}
    \]
    for any nonnegative integer $n$
    If we write it additively, then
    \[
        na = \underbrace{a + a + \cdots + a}_{n\text{ times}}.
    \]
\end{remark}

\begin{example}
    \begin{enumerate}[(a)]
        \item $G = \Z$ becomes a group when we equip it with the operation of addition (the identity is 0 and the inverse of $a$ is $-a$).
        However, $\Z$ is not a group when we equip it with multiplication since all integers not equal to $\pm 1$ lack multiplicative inverses.

        \item $\Z/n\Z$ is a group when we equip it with addition modulo $n$.
        It's a group with respect to multiplication (when we throw out 0 of course) if and only if $n$ is prime.
        In general, to get a group with respect to multiplication, we have to look at $(\Z/n\Z)^\times$, the (equivalence classes of) integers coprime to $n$.

        \item The set of $n\times n$ matrices with real (or complex, or $\F_p$) entries and nonzero determinant is a group with respect to matrix multiplication.
        We denote this group by $\text{GL}_n(\R)$ (or $\text{GL}_n(\C)$ or $\text{GL}_n(\F_p)$), the \emph{general linear group}.

        \item If $X$ is a set, then the set of all bijections from $X$ to itself, denoted by $S_X$, is a group with respect to the operation of function composition (what is the identity? how do you invert?)
    \end{enumerate}   
\end{example}


Lots of the algebra that we've built up for $\Z/n\Z$ (and lots of the cryptography) carries over to arbitrary groups.
In particular, we still have the notion of an element's order.

\begin{definition}
    Let $G$ be a group and let $a\in G$.
    The smallest positive integer $d$ such that $a^d = e$ is called the \emph{order} of $a$.
    If there is no such integer, then we say $a$ has \emph{infinite order}.
\end{definition}

We have a similar interplay between order and divisibility.

\begin{proposition}
    Let $G$ be a finite group (i.e., $|G|$ is finite).
    Then every element of $G$ has finite order.
    Moreover, if $a\in G$ has order $d$ and if $a^k = e$, then $d \mid k$.
\end{proposition}

\begin{proof}
    We look at the powers of $a$:
    \[
        a, a^1, a^2, \ldots
    \]
    If these elements of $G$ were all distinct, then $G$ wouldn't be a \emph{finite} group, so we must have repetitions here.
    Say $a^i = a^j$ for some $i<j$.
    Then we can multiply both sides by the inverse of $a$ $i$ times to get $a^{j-i} = e$.
    Since $j-i$ is positive, we've found a positive integer power of $a$ that gives us $e$.
    Let $d$ be the smallest such power (the order of $a$ - this comes from the \emph{well-ordered property} of the natural numbers).


    Now suppose that $a^k = e$ for some positive $k$.
    Divide $k$ by $d$ with remainder to get
    \[
        k = dq + r,\quad \text{with }0\leq r < d.
    \]
    Then 
    \[
        e = a^k = a^{dq + r} = (a^d)^q\cdot a^r = 1\cdot a^r = a^r.
    \]
    Since $d$ is the smallest positive power of $a$ equal to $e$, we must have that $r = 0$ and $d\mid k$.
\end{proof}


We also have a version of Fermat's little theorem that works in any finite group.

\begin{theorem}[Lagrange's Theorem]
    Let $G$ be a finite group with $n$ elements.
    Then $a^n  = e$ for all $a\in G$.
    In particular, the order of $a$ divides $n$.
\end{theorem}
\begin{proof}
    This theorem really is true for \emph{any} finite group, but since most of the groups used in cryptography are commutative, you'll prove the theorem under the additional assumption that $G$ is commutative in your discussion section.
    The proof of the more general statement isn't very complicated and it's one of the first things you learn in Math 120A.
\end{proof}










\subsection{Order Notation and Algorithms}
In cryptography we often find ourselves concerned with algorithms (like when we think about how to compute things or when we think about what an adversary needs to do in order to compromise or communications).
More specifically, we're usually interested in how ``good'' and algorithm is, usually measured in the number of ``steps'' it takes to complete (may the number of times we perform a basic arithmetic operation like multiplication or addition or maybe the number of times we have to query an oracle).


\begin{example}
    Consider the problem of finding your name on an alphabetized roster containing $n$ names.
    One way we can do this is by reading the names off one by one until we find your name.
    In the worst case, your name is the last on the list and we have to read through all $n$ names.

    Another approach is to look at the name in the middle of the list (if there is an even number of names, just pick one of the two in the ``middle'' arbitrarily.
    If your name comes before this one in the alphabet, then we can safely throw out all the names that come after this middle name (and vice-versa if your name comes after it; if we happened to pick your name at this step, then we're done!).
    This just leaves $n/2$ names to look at.
    Now do the same thing - look at the name in the middle of this list then throw out everything after it if your name comes before this name (and throw out everything before if your name comes after).

    How many steps does this take to run?
    At each step we cut the number of things we need to search through in half.
    We keep doing this until we find your name.
    At worst, we have to check $T$ names, where $T$ is the smallest integer such that $n/2^T < 1$.
    Therefore, this algorithm requires at most $\lfloor \log_2(n)\rfloor + 1$ steps to complete.
\end{example}

\begin{remark}
    In this example, we bounded the number of steps it took to complete our task by considering the \emph{worst case}.
    When counting the number of steps, we always assume we're working in the worst case (unless we specify otherwise).
\end{remark}

The second algorithm in the above example, although slightly more complicated than the first,
is way more efficient (again, in the worst case scenario).
One way to intuit this is to compare the growth of the functions $n$ and $\log_2 n$ -- the former grows much more rapidly than the latter (if the list had 1000 names on it, then it would take at most ten steps to find your name!)
This motivates the following framework for comparing the growth of functions.

\begin{definition}
    Let $f, g$ be functions $\N\to \R$.
    We say that $f(n) = O(g(n))$, read ``$f$ is big $O$ of $g$'' if there exist positive constants $c$ and $C$ such that
    \[
        f(x) \leq Cg(x)
    \]
    for all $x \geq c$.
    That is, $f$ is \emph{eventually} bounded by some constant multiple of $g$.
\end{definition}


\begin{example}
    If $f(n) = 3n^2 + 5\log n$, then $f(n) = O(n^2)$.
    To see this, note that $\log n$ is smaller than $n$ for any $n \geq 1$.
    Likewise, $n \leq n^2$ for all $n \geq 1$, so we have
    \begin{align*}
        3n^2 + 5\log n &\leq 3n^2 + 5n \qquad \text{if }n \geq 1\\
        &\leq 8n^2 \qquad \text{if }n\geq 1.
    \end{align*}
\end{example}


The following proposition gives an easier way of checking if $f(n) = O(g(n))$.
\begin{proposition}
    If
    \[
        \lim_{n\to \infty}\frac{f(n)}{g(n)} = L < \infty,
    \]
    then $f(n) = O(g(n))$.
\end{proposition}
\begin{proof}
    By the definition of limits, we must have that
    \[
        \left|\frac{f(n)}{g(n)} - L\right| \leq 1
    \]
    for all $n$ greater than some $c$.
    Consequently, when $n \geq c$, we have that
    \[
        f(n) \leq (L+1)g(n),
    \]
    so setting $C = L+1$ in the definition of big $O$ notation proves the claim.
\end{proof}


Let's record some basic facts and examples.
\begin{example}
    \begin{enumerate}
        \item (We can ignore constant factors) For any function $g(n): \N\to \R$ we have that $Kg(n) = O(g(n))$ for any constant $K$.

        \item $\log n = O(n^\alpha)$ for any $\alpha > 0$. That is, $\log n$ grows more slowly than \emph{any} polynomial.
        You can prove this with L'Hopital's rule.

        \item If $f(n)$ is any polynomial, then $f(n) = O(2^n)$.

        \item If $f(n)$ is a constant, then $f(n) = O(1)$.

        \item If $f(n)$ is the number of steps it takes to find a name in an alphabetized roster, then $f(n) = O(\log n)$.

        \item If $f(a,b)$ is the number of steps it takes to compute the greatest common divisor of $a$ and $b$, then $f(a,b) = O(\log b)$, where $b \geq a$.
    \end{enumerate}
\end{example}


\begin{remark}
    Usually we care about how the number of steps it takes an algorithm to complete scales with the size of its input.
    The size of the input is often measured in the number of bits it takes to store it (this is how computers store things).
    With this in mind, if the input to a problem is $k$ bits long and it takes an algorithm $O(k^A)$ steps to complete for some constant $A \geq 0$, then we say that the algorithm runs in \emph{polynomial time}.
    If it takes $O(2^{ck})$ steps for some $c \geq 0$, then we say it takes \emph{exponential time}.

    Between these two classes we have \emph{subexponential time}, where the number of steps is $O(2^{\epsilon k})$ for every $\epsilon > 0$ (here the constants in the definition of big $O$ notation are allowed to depend on $\epsilon$).
    Generally speaking we consider polynomial-time algorithms to be efficient (they usually run decently well on computers if the degree of the polynomial isn't too big) and exponential time algorithms to be inefficient.
\end{remark}


\begin{example}
    The brute-force algorithm for solving the discrete logarithm problem in $\F_p^\times$ requires checking at most $p-1$ different values of $g^x$.
    Since $p$ takes $k \approx \log_2 p$ bits to store, this algorithm runs in exponential time.
    We'll explore better algorithms later.
\end{example}

\begin{example}
    In some groups the discrete logarithm problem is extremely easy.
    For example, if we consider the DLP in the \emph{additive} group $\F_p$, then we want to find $x$ such that $x\cdot g \equiv h\pmod p$ for fixed $g$ and $h$.
    But we can do this by just computing $g^{-1}\pmod p$ and multiplying, which takes $O(\log p)$ steps with the Euclidean algorithm.

    Finding groups where the discrete logarithm problem is hard is important in cryptography.
\end{example}










\subsection{A collision algorithm for the DLP}
Recall that the discrete logarithm problem (DLP) in a group $G$ asks us to find an integer $x$ such that
\[
    g^x = h
\]
for some fixed $g$ and $h$ in $G$, if such an $x$ exists.
If $G$ has $N$ elements, then $g^N = e$ by Lagrange's theorem.
We can then solve the DLP by simply trying all powers of $g$,
\[
    g,\ g^2,\ g^3,\ldots, g^N
\]
until we get $h$ (or we never get $h$ and the DLP has no solution).
This takes $O(N)$ steps and requires $O(1)$ storage since we only need to store one power of $g$ at a time when performing this algorithm.

However, an element of $G$ takes $b := \log N$ bits to store or specify, so this algorithm runs in $O(2^b)$ steps -- exponential in the size of the input.


Let's illustrate a new algorithm with an example.
Let's just say that $G$ has $N = 100$ elements.
%The discrete logarithm of $h$ is in the range $1\leq x \leq 100$, and we can write $x = 10q + r$, where $q,r < 10$.
Now consider the lists
\[
    A = \{e, g^1, g^2, \ldots, g^9\},\qquad B = \{h,h\cdot g^{-10}, h\cdot g^{-20},\ldots, h\cdot g^{-90} \}.
\]
If these lists have a common element, then
\[
    g^a = h\cdot g^{-b\cdot 10} \implies g^{b\cdot 10 + a} = h,
\]
and we've found $\log_g(h)$.
If the DLP has a solution for this $g$ and $h$, then there must be a match between these two lists.
If $x$ is the solution, write $x = 10q + r$ where $q,r<10$.
Then $g^r$ is in $A$ and $h\cdot g^{-10q}$ is in $B$.
\begin{remark}
    An important part of this algorithm is finding a match between two lists.
    If the lists $A$ and $B$ have $t$ elements each and they have a match between them, then we can find it in $O(t\log t)$ steps.
    This comes from the fact that the lists can both be sorted in $O(t \log t)$ steps (say, by using merge sort).
    Then go through each of the $t$ elements of $A$ and take at most $O(\log t)$ steps to find it in the sorted list $B$.
\end{remark}

In general, suppose we want to solve $g^x = h$ in the group $G$, which has $N$ elements.
First let $n = 1 + \lfloor \sqrt{N}\rfloor$.
Then do the following.
\begin{enumerate}
    \item Create the two lists
    \[
        A = \{e, g, g^2, \ldots, g^n\},\qquad B = \{h, h\cdot g^{-n}, h\cdot g^{-2n}, \ldots, h\cdot g^{-n^2}\}.
    \]

    \item Sort the lists $A$ and $B$

    \item Find a match between the lists.

    \item The match is $g^i = h\cdot g^{-jn}$. Return $x = i +jn$ as a solution to $g^x = h$.
\end{enumerate}

How many steps does this algorithm take?
\begin{enumerate}
\item The elements of $A$ and $B$ can both be generated by just repeatedly multiplying by a single element (by $g$ in the list $A$ and by $g^{-n}$ in list $B$), so $O(n) = O(\sqrt{N})$ steps for step 1.

\item We've already remarked that sorting the lists $A$ and $B$ takes $O(n\log n) = O(\sqrt{N}\log N)$ steps.

\item Once the lists are sorted, it takes $O(\log n) = O(\log N)$ steps per element of $A$ to see if it has a match in $B$, so $O(n\log n) = O(\sqrt{N}\log N)$ steps here.
\end{enumerate}
Thus, the total number of steps is
\[
    O(\sqrt{N} + \sqrt{N}\log N + \sqrt{N}\log N) = O(\sqrt{N}\log N).
\]
If the DLP has a solution, $x$, then the two lists must have a match.
Write $x = nq + r$ with $0\leq r < n$ and $q \geq 0$.
Then $g^r$ is in list $A$ and $h\cdot g^{-nq}$ is on list $B$ so long as $q\leq n$.
If $q > n$, then we would have $qn > n^2 > N$.
But $r \geq 0$, so we would have $x = nq + r > N$, which contradicts the fact that, if a solution exists, it must be at most $N$.

This algorithm is sometimes called \emph{baby step giant step}. The elements of $A$ are the ``baby steps'' and those of $B$ are the ``giant steps''.

\begin{remark}
    While this algorithm is definitely better than brute-force -- $O(\sqrt{N}\log N)$ instead of $O(N)$, it still runs in time that is exponential in the size of the input.
    Indeed, since and element of $N$ takes $b = \log N$ bits to specify, this algorithm runs in time $O(2^{b/2}\cdot b)$.
\end{remark}










\subsection{The Chinese Remainder Theorem}
Consider the discrete logarithm problem in $\F_p^\times$.
That is, we want to find a solution $x$ to the equation
\[
    g^x \equiv h\pmod p.
\]
By Fermat's little theorem, the solution to the above equation, if it exists, is defined modulo $p-1$.
Now a big idea in computer science is to take a problem and try to decompose it into smaller problems that are (hopefully) easier to solve.
Which smaller problems should we break the DLP into?
Well if we can factor $p-1$ into smaller primes,
\[
    p-1 = q_1^{e_1}q_2^{e_2}\cdots q_t^{e_t},
\]
then maybe we can find what $x$ is modulo each $q_i^{e_i}$ first and then somehow weave these solutions together into a solution modulo $p-1$.
Let's illustrate this with a simple example.

\begin{example}
    Let's find an integer $x$ that solves both of the congruences
    \[
        x \equiv 2 \pmod 5\qquad\text{and}\qquad x\equiv 3\pmod 7.
    \]
    The full set of integer solutions to the first equation is
    \begin{equation}\label{one congruence}
        x = 2 + 5k,\quad k\in \Z.
    \end{equation}
    If we substitute this into the second congruence we get
    \[
        2 + 5k \equiv 3\pmod 7 \implies 5k\equiv 1\pmod 7.
    \]
    Now we simply multiply both sides by the inverse of 5 modulo 7.
    This can indeed be done since $\gcd(5,7) = 1$.
    We can even do it efficiently using the Euclidean algorithm (or just guess-and-check here since the numbers are small).
    In any case, the inverse of 5 modulo 7 is 3, so we have
    \[
        k \equiv 3\pmod 7.
    \]
    That is, $k = 3 + 7k'$ for any integer $k'$.

    If we substitute this into (\ref{one congruence}), we see that
    \[
        2 + 5(3 + 7k) = 17 + 35k'
    \]
    is a full set of solutions to both congruences.
    In other words, the solution is $x\equiv 17\pmod{35}$.
\end{example}

This example gives an algorithm for how to prove the following theorem.

\begin{theorem}[Chinese Remainder Theorem]
    Let $m_1, m_2, \ldots, m_k$ be a collection of pairwise coprime integers, i.e.
    \[
        \gcd(m_i, m_j) = 1\quad\text{for }i\neq j.
    \]
    Let $a_1, a_2, \ldots, a_k$ be arbitrary integers.
    Then the system of simultaneous congruences
    \[
        x \equiv a_1\pmod{m_1},\ x\equiv a_2\pmod{m_2},\quad \ldots,\quad x \equiv a_k\pmod{m_k}
    \]
    has a unique solution modulo $m_1m_2\ldots m_k$.
\end{theorem}

\begin{proof}
    We explicitly construct the solution in a neat and algorithmic way.
    Suppose we already have an integer solution $x = c_i$ the the first $i$ congruences.
    \[
        x \equiv a_1\pmod{m_1},\ x\equiv a_2\pmod{m_2},\quad \ldots,\quad x \equiv a_i\pmod{m_i}
    \]
    For example, $c_1 = a_1$ is a solution to the first congruence.
    Now let's show how to extend this solution to solve one more congruence.
    Look at the integers of the form
    \[
        x = c_1 + m_1m_2\cdots m_iy
    \]
    as $y$ ranges over the integers.
    If we reduce such an $x$ modulo $m_j$ for any $j\leq i$, we obtain another solution to the first $i$ congruences, so we just need to pick $y$ so that $x\equiv a_{i+1}\pmod{m_{i+1}}$.
    That is, we need to solve the equation
    \[
        c_i + m_1m_2\cdots m_i y \equiv a_{i+1}\pmod{m_{i+1}}.
    \]
    But we can do this by just moving the $c_i$ over and inverting $m_1\cdots m_i$ modulo $m_{i+1}$.
    We can indeed do this because the $m_j$'s are all pairwise coprime, so
    \[
        \gcd(m_1m_2\cdots m_i, m_{i+1}) = 1.
    \]
    You'll prove the uniqueness of this solution on your homework.
\end{proof}











\subsection{The Pohlig-Hellman Algorithm}
We hope to find an easier way to solve the discrete logarithm problem (find $x$ such that $g^x \equiv h\pmod p$).
Our idea was to factor $p-1 = q_1^{e_1}\cdots q_t^{e_t}$ and then somehow find $x\pmod{p_i^{e_i}}$ for each $i$.
The Chinese remainder theorem then gives us a way to (efficiently) weave these solutions together into $x\pmod{p-1}$.
This is essentially the idea behind the Pohlig-Hellman algorithm.


\begin{theorem}
    Let $G$ be a group and suppose we have an algorithm to solve the discrete logarithm problem in $G$ for any element whose order is a prime power.
    That is, if $g\in G$ has order $q^e$, suppose we can solve $g^x = h$ in $O(S_{q^e})$ steps.

    Then if $g\in G$ has order $N$, where $N$ has prime factorization
    \[
        N = q_1^{e_1}q_2^{e_2}\cdots q_t^{e_t},
    \]
    the discrete logarithm problem $g^x = h$ can be solved in
    \[
        O\left(\sum_{i=1}^tS_{q_i^{e_i}} + \log N\right)
    \]
    steps with the following procedure.
    \begin{enumerate}
        \item For each $1\leq i \leq t$, let
        \[
            g_i = g^{N/q_i^{e_i}}\qquad\text{and}\qquad h_i = h^{N/q_i^{e_i}}.
        \]
        Solve the discrete logarithm problem
        \[
            g_i^y = h_i
        \]
        and let $y_i$ be a solution.

        \item Use the Chinese remainder theorem to solve
        \begin{equation}\label{simul}
            x \equiv y_1\pmod{q_1^{e_1}},\ \ldots,\ x \equiv y_t\pmod{q_t^{e_t}}.
        \end{equation}
    \end{enumerate}
\end{theorem}

\begin{proof}
    Since $g$ has order $N$, $g_i = g^{N/q_i^{e_i}}$ has order $q_i^{e_i}$ (why?).
    Consequently, the first step of our algorithm takes $O(\sum S_{q_i^{e_i}})$ steps.
    The Chinese remainder theorem step consists of just basic addition and inversion modulo $q_i^{e_i}$.
    Since inversion can be done in $O(\log q_i^{e_i}) = O(\log N)$ steps, the entirety of step 2 takes just $O(t \log N) = O(\log N)$ steps.

    Since $x$ satisfies the simultaneous congruences (\ref{simul}), we can write
    \[
        x = y_i + z_iq_i^{e_i}
    \]
    for some integer $z_i$.
    We then have
    \begin{align*}
        (g^x)^{N/q_e^{e_i}} &= (g^{y_i + z_iq_i^{e_i}})^{N/q_i^{e_i}}\\
        &= (g^{N/q_i^{e_i}})^{y_i}\cdot g^{Nz_i}\\
        &= g_i^{y_i}\\
        &= h_i\\
        &= h^{N/q_i^{e_i}}.
    \end{align*}
    The third line follows from the fact that $a^N = e$ for any $a\in G$ by Lagrange's theorem.
    If we take the logarithm to the base $g$ of both sides of this, we have
    \begin{equation}\label{coprime}
        \frac{N}{q_i^{e_i}}x \equiv \frac{N}{q_i^{e_i}}\log_g(h)\pmod N.
    \end{equation}
    Since the integers $N/q_1^{e_1}, \ldots, N/q_t^{e_t}$ are pairwise coprime, we can find integers $c_1, \ldots, c_t$ such that
    \[
        \frac{N}{q_1^{e_1}}c_1 +\cdots + \frac{N}{q_t^{e_t}}c_t  =1.
    \]
    If we multiply both sides of (\ref{coprime}) by $c_i$ and sum over $i$, then we see that
    \begin{align*}
        x & \equiv  \sum_{i=1}^tx\cdot\frac{N}{q_i^{e_i}}c_i\\
        &\equiv \sum_{i=1}^t\log_g(h)\cdot\frac{N}{q_i^{e_i}}c_i\\
        &\equiv h\pmod N.
    \end{align*}
\end{proof}

We've reduced the discrete logarithm problem to the case where the base has prime power order.
Now we show how to reduce the problem even further to just a base with prime order.

\begin{proposition}
    Let $G$ be a group, $q$ a prime and suppose we have an algorithm that takes $S_q$ steps to solve the DLP $g^x = h$ in $G$ whenever $g$ has order $q$.
    Then if $g$ has order $q^e$, we can solve the discrete log problem $g^x = h$ in $O(eS_q)$ steps.
\end{proposition}

\begin{proof}
    The idea is to write the unknown exponent $x$ in base $q$.
    \[
        x = x_0 + x_1q + x_2q^2 + \cdots + x_{e-1}q^{e-1}\quad\text{with }0\leq x_i<q.
    \]
    We have
    \begin{align*}
        h^{q^{e-1}} &= (g^x)^{q^{e-1}}\\
        &= \left(g^{x_0 + x_1q + \cdots + x_{e-1}q^{e-1}}\right)^{q^{e-1}}\\
        &= (g^{q^{e-1}})^{x_0}\cdot \left(g^{q^e}\right)^{x_1 + x_2q + \cdots + x_{e-1}q^{e-2}}\\
        &= \left(g^{q^{e-1}}\right)^{x_0}.
    \end{align*}
    The last line follows since $g^{q^{e}} = 1$.
    Since $g^{q^{e-1}}$ has order $q$, this is a problem we can solve for $x_0$ in $O(S_q)$ steps.
    We essentially repeat this to find $x_2$.
    Write
    \begin{align*}
        h^{q^{e-2}} &= (g^x)^{q^{e-2}}\\
        &= \left(g^{x_0 + x_1q + \cdots + x_{e-1}q^{e-1}}\right)^{q^{e-2}}\\
        &= g^{x_0q^{e-2}}\cdot \left(g^{q^{e-1}}\right)^{x_1}\cdot  \left(g^{q^e}\right)^{x_2 + x_3q + \cdots + x_{e-1}q^{e-3}}\\
        &= g^{x_0q^{e-2}}\cdot \left(g^{q^{e-1}}\right)^{x_1}.
    \end{align*}
    But we know $x_0$, so we can move it over to the LHS to get
    \[
        (h\cdot g^{-x_0})^{q^{e-2}} = (g^{q^{e-1}})^{x_1},
    \]
    another DLP with base of order $q$.
    We then solve this problem in $O(S_q)$ steps to find $x_1$.
    In general, to get $x_i$, we solve the problem
    \[
        (g^{q^{e-1}})^{x_i} = \left(h\cdot g^{-x_0 - x_1q - \cdots - x_{i-1}q^{i-1}}\right)^{q^{e-i-1}}.
    \]
    In total, this takes $O(eS_q)$ steps.
\end{proof}

The moral of the story is that if $p-1$ factors into many small primes, then then the Pohlig-Hellman algorithm gives a viable attack on the DLP.






\begin{thebibliography}{13}

\bibitem{HPS} Hoffstein, Jeffrey, Jill Pipher and Joseph H. Silverman. \href{https://link.springer.com/book/10.1007/978-1-4939-1711-2}{\textit{An Introduction to Mathematical Cryptography}}. Second Edition. Springer New York, NY. 2014.
% \bibitem{Grav} Gravner, Janko. Online lecture notes, https://www.math.ucdavis.edu/~gravner/MAT135A/resources/lecturenotes.pdf

% \bibitem{Prob and Comp} Mitzenmacher, Michael, and Eli Upfal. Probability and computing: Randomization and probabilistic techniques in algorithms and data analysis. Cambridge university press, 2017.
% \bibitem{Ross} Ross, Sheldon M. A first course in probability. Vol. 7. Upper Saddle River, NJ: Pearson Prentice Hall, 2006.


 
\end{thebibliography}


\end{document}