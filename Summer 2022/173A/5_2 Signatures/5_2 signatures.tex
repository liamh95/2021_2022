\documentclass[11pt,letterpaper]{article}
\usepackage{amssymb,amsfonts,color,graphicx,amsmath,enumerate}
\usepackage{tikz}
\usepackage{amsthm}

\newcommand{\naturals}{\mathbb{N}}
\newcommand{\integers}{\mathbb{Z}}
\newcommand{\complex}{\mathbb{C}}
\newcommand{\reals}{\mathbb{R}}
\newcommand{\mcal}[1]{\mathcal{#1}}
\newcommand{\rationals}{\mathbb{Q}}
\newcommand{\Lp}[2]{\left\|{#1}\right\|_{L^{#2}}}
\newcommand{\F}{\mathbb{F}}
\newcommand{\affine}{\mathbb{A}}
\newcommand{\E}{\mathbb{E}}
\newcommand{\Prob}{\mathbb{P}}
\newcommand{\Var}{\text{Var}}
\newcommand{\ind}{\mathbbm{1}}
\newcommand{\Cov}{\text{Cov}}

\newenvironment{solution}
{\begin{proof}[Solution]}
{\end{proof}}

\voffset=-3cm
\hoffset=-2.25cm
\textheight=24cm
\textwidth=17.25cm
\addtolength{\jot}{8pt}
\linespread{1.3}

\begin{document}
\begin{center}
{\bf \Large Math 173A - Signatures}
\vspace{0.2cm}
\hrule
\end{center}

% why 3N and not 2N?

\begin{enumerate}

    \item Sign a ``document'' with RSA and have your partner verify the signature.
    That is, pick two primes $p$ and $q$ and a verification key $e$ coprime to $(p-1)(q-1)$.
    Publish $N  = pq$ and $e$.
    Compute your signing key $d$ with $de\equiv 1 \pmod{(p-1)(q-1)}$ and keep it secret.

    Now choose a document $D \pmod N$ and sign it by computing $S \equiv D^d\pmod N$.
    Have your partner verify your signature by computing $S^e\pmod N$ and making sure it's congruent to $D$.


    \item Sign a ``document'' with ElGamal and have your partner verify the signature.
    That is, pick a prime $p$ and a primitive root $g$ modulo $p$.
    Choose a secret signing key $1\leq a\leq p-1$ and publish the verification key $A \equiv g^a\pmod p$.

    Now choose a document $D \pmod{p-1}$ and sign it.
    Do this by picking a random $1 < k < p$ with $\gcd(k, p-1) = 1$.
    Compute the signature
    \[
        S_1 \equiv g^k\pmod p\qquad \text{and}\qquad S_2 \equiv (D-aS_1)k^{-1}\pmod {p-1}.
    \]

    Have your partner verify your signature by verifying that $A^{S_1}S_1^{S_2}\equiv g^D\pmod p$.
\end{enumerate}
\vfill
Here are some primes for RSA
\begin{multline*}
    23,\ 29,\ 31,\ 37,\ 41,\ 43,\ 47,\ 53,\ 59,\ 61,\ 67,\ 71,\ 73,\ 79,\ 83,\ 89,\ 97.
\end{multline*}

\vfill

Here are some primes and primitive roots for ElGamal.

\begin{center}
    \begin{tabular}{|c|c|}
        \hline
        $p$ & Primitive root modulo $p$\\
        \hline
        241 & 7\\
        353 & 3\\
        419 & 2\\
        557 & 2\\
        683 & 5\\
        \hline
    \end{tabular}
\end{center}

\vfill

\end{document}