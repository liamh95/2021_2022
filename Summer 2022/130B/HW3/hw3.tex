\documentclass[11pt,letterpaper]{article}
\usepackage{amssymb,amsfonts,color,graphicx,amsmath,enumerate}
\usepackage{tikz}
\usepackage{amsthm}

\newcommand{\naturals}{\mathbb{N}}
\newcommand{\integers}{\mathbb{Z}}
\newcommand{\complex}{\mathbb{C}}
\newcommand{\reals}{\mathbb{R}}
\newcommand{\mcal}[1]{\mathcal{#1}}
\newcommand{\rationals}{\mathbb{Q}}
\newcommand{\Lp}[2]{\left\|{#1}\right\|_{L^{#2}}}
\newcommand{\F}{\mathbb{F}}
\newcommand{\affine}{\mathbb{A}}
\newcommand{\E}{\mathbb{E}}
\newcommand{\Prob}{\mathbb{P}}
\newcommand{\Var}{\text{Var}}
\newcommand{\ind}{\mathbbm{1}}
\newcommand{\Cov}{\text{Cov}}

\newenvironment{solution}
{\begin{proof}[Solution]}
{\end{proof}}

\voffset=-3cm
\hoffset=-2.25cm
\textheight=24cm
\textwidth=17.25cm
\addtolength{\jot}{8pt}
\linespread{1.3}

\begin{document}
\begin{center}
{\bf \Large Math 130B - Homework 3}
\vspace{0.2cm}
\hrule
\end{center}


% method of conditional expectations max cut or ramsey
% some of the conditional expectation prediction stuff

\begin{enumerate}

    \item $n$ people arrive separately to a professional dinner.
    Upon arrival, each person looks to see if he or she has any friends among those present.
    That person then sits either at the table of a friend or at an unoccupied table if none of those present is a friend.
    Assuming that each of the $\binom{n}{2}$ pairs of people is, independently, a pair of friends with probability $p$, find the expected number of occupied tables.

    \item If $X_1, X_2, \ldots, X_n$ are independent and identically distributed random variables having uniform distribution on $(0,1)$, find
    \begin{enumerate}
        \item $E[\max(X_1, \ldots, X_n)]$;
        \item $E[\min(X_1, \ldots, X_n)]$.
    \end{enumerate}

    \item The random variables $X$ and $Y$ have a joint density function given by
    \[
        f(x,y) = 2e^{-2x}/x,\qquad 0\leq x < \infty,\ 0\leq y\leq x.
    \]
    Compute $\Cov(X,Y)$.


    \item Consider the following dice game.
    Players 1 and 2 each roll a six-sided die.
    The casino then rolls a die to determine the outcome according to the following rule:
    Player $i$, where $i = 1,2$, wins if their roll is strictly greater than the casino's.
    For $i = 1,2$, let $X_i$ be the random variable indicating whether or not player $i$ wins this game.

    Are the variables $X_1$ and $X_2$ positively correlated, negatively correlated, or uncorrelated?
    Give both a rigorous justification and an intuitive one.


    \item Let $X$ and $Y$ be identically distributed, but not necessarily independent random variables.
    Show that $X+Y$ and $X-Y$ are uncorrelated.

    \item Let $X$ be a normal random variable with mean $\mu_X$ and variance $\sigma_X^2$.
    Also let $Z$ be another normal random variable, independent of $X$, with mean 0 and variance $\sigma_Z^2$.
    Find the correlation between $X$ and $Y$, where
    \[
        Y = a + bX + Z.
    \]
    \emph{Think of $a+bX$ as being the ``signal,'' and $Z$ the ``noise'' that gets added on top.}


    \item There are two coins in a box; their probabilities for landing on heads when they are flipped are .4 and .7, respectively. One of the coins is randomly chosen and flipped 10 times.
    Given that two of the first three flips landed on heads, what is the conditional expected number of heads in the 10 flips.


    \item You run a whale-watching business in San Diego.
    Every day you are unable to operate your tour due to bad weather with probability $p$, independently of all other days.
    You work every day except the bad-weather days.

    Let $Y$ be the number of consecutive days you work between bad-weather days and let $X$ be the total number of customers who attend your tour in this period of $Y$ days.
    Conditional on $Y$, suppose the distribution of $X$ is $\text{Pois}(\mu Y)$ for some positive $\mu$.

    \begin{enumerate}
        \item What kind of random variable is $Y$? What are its expectation and variance?

        \item Find the expectation and variance of the number of customers you see between bad weather days.
    \end{enumerate}



    \item Remember that if $X$ is a random variable and $E[X] \geq a$, then $X\geq a$ with positive probability.
    This is great, but how can we actually \emph{find} an instance of $X$ where $X\geq a$?

    \begin{enumerate}[(a)]
        \item Suppose $X$ is a random variable taking values in $\{1, 2, \ldots, k\}$ and that $f$ is some function with $E[f(X)] \geq a$.
        One way to find an $x$ with $1\leq x\leq k$ and $f(x) \geq a$ is to just try all possibilities for $x$ and plug them into $f$.
        How many times would you need to evaluate $f$ to use this method?

        \item Say $X_1, \ldots, X_n$ are independent random variables, each taking values $\{1, 2, \ldots, k\}$ and now $f$ is some function with $E[f(X_1, \ldots, X_n)] \geq a$.
        Using the same idea from part (a), how many evaluations of $f$ would you need in order to find $x_1, \ldots, x_n$ with $f(x_1, \ldots, x_n) \geq a$?
    \end{enumerate}

    Here's another strategy for finding $(x_1, \ldots, x_n)$ with $f(x_1, \ldots, x_n) \geq a$.

    \begin{enumerate}[(a)]
        \setcounter{enumii}{2}
        \item Set $g_1(x)$ to be the conditional expectation
        \[
            g_1(x) = E[f(X_1, \ldots, X_n) \mid X_1 = x].
        \]
        Show that $E_{X_1}[g_1(X_1)] \geq a$ and determine how many evaluations of $g_1$ you need in order to find $x_1$ with $g(x_1) \geq a$.

        \item Now let $g_2(x)$ be the conditional expectation
        \[
            g_2(x) = E[f(X_1, \ldots, X_n) \mid X_1 = x_1, X_2 = x],
        \]
        where $x_1$ is from part (c).
        Argue that $E[g_2(X_2)] \geq a$ and determine how many evaluations of $g_2$ you need to find $x_2$ with $g(x_2) \geq a$.

        \item Repeat the above process $n$ times.
        On the $i$-th step, define the conditional expectation
        \[
            g_i(x) = E[f(X_1, \ldots, X_n) \mid X_1 = x_1, \ldots, X_{i-1} = x_{i-1}, X_i = x],
        \]
        where $x_j$ is such that $g_j(x_j) \geq a$ for each $j < i$.
        Argue that $x_1, \ldots x_n$ satisfy $f(x_1, \ldots, x_n) \geq a$.
        How many \emph{total} steps does it take to find $x_1, x_2, \ldots, x_n$?
        Compare this to your answer in part (b).
    \end{enumerate}



\end{enumerate}

\end{document}