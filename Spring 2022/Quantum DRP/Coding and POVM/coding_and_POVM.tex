\documentclass[12pt]{report}  

\pagestyle{empty}

\usepackage{graphics}
\usepackage{amsmath,amssymb,amsthm, multicol,array}
\usepackage[pdftex]{graphicx}
\usepackage{enumerate}
\usepackage{epsf}

\theoremstyle{definition}
\newtheorem{thm}{Theorem}
\newtheorem{claim}{Claim}
\newtheorem{lem}[thm]{Lemma}
\newtheorem{cor}[thm]{Corollary}
\newtheorem{rem}[thm]{Remark}
\newtheorem{remark}[thm]{Remark}
\newtheorem{conj}[thm]{Conjecture}
\newtheorem{definition}[thm]{Definition}
\newtheorem{question}{Question}

\newcommand{\naturals}{\mathbb{N}}
\newcommand{\integers}{\mathbb{Z}}
\newcommand{\complex}{\mathbb{C}}
\newcommand{\reals}{\mathbb{R}}
\newcommand{\mcal}[1]{\mathcal{#1}}
\newcommand{\rationals}{\mathbb{Q}}
\newcommand{\Aut}{\text{Aut}}
\newcommand{\Lp}[2]{\left\|{#1}\right\|_{L^{#2}}}
\newcommand{\tr}{\text{tr}}
\newcommand{\field}{\mathbb{F}}
\newcommand{\ket}[1]{\left|{#1}\right\rangle}
\newcommand{\bra}[1]{\left\langle {#1}\right|}

\addtolength{\oddsidemargin}{-.75in}
\addtolength{\evensidemargin}{-.75in}
\addtolength{\textwidth}{1.5in}
\addtolength{\topmargin}{-1in}
\addtolength{\textheight}{2.25in}

\begin{document}
\begin{center}
{\bf \Large Math 13 - Week 9: Equivalence Relations}
\vspace{0.2cm}
\hrule
\end{center}

Let's review the superdense coding protocol from section 2.3.
Here, Alice wants to send a 2-bit message to Bob.
They start by preparing two qubits in the following entangled state,
\begin{equation}
	\ket{\psi} = \frac{\ket{00} + \ket{11}}{\sqrt{2}}.
\end{equation}
Alice takes the first qubit and Bob the second.
Now if Alice wants to send Bob the message ``00'', she does nothing to her qubit.
If she wants to send ``01'' she applies $Z$, for ``10'' she applies $X$ and for ``11'' she applies $iY$.
\begin{question}
	Aren't $I$, $X$, $Y$ and $Z$ operators on \emph{single} qubits? If so, how can Alice apply them to the \emph{two} qubit state, $\ket{\psi}$?
\end{question}
\begin{proof}[Solution]
	Alice isn't applying these single qubit operators to the \emph{whole} state, but to just her single qubit.
	If she were to apply the operator $X$ to just her qubit (assume it's the first one), then the \emph{whole} state becomes
	\[
		(X\otimes I)\ket{\psi}.
	\]
	The $I$ above represents Bob doing nothing to his qubit.
\end{proof}

Alice's actions are encoded in the following display.
\begin{equation}
\begin{split}
	00\ : \ket{\psi} & \mapsto (I\otimes I)\ket{\psi} = \frac{\ket{00} + \ket{11}}{\sqrt{2}}\\
	01\ : \ket{\psi} & \mapsto (Z\otimes I)\ket{\psi} = \frac{\ket{00} - \ket{11}}{\sqrt{2}}\\
	10\ : \ket{\psi} & \mapsto (X\otimes I)\ket{\psi} = \frac{\ket{10} + \ket{01}}{\sqrt{2}}\\
	11\ : \ket{\psi} & \mapsto (iY\otimes I)\ket{\psi}= \frac{\ket{10} - \ket{10}}{\sqrt{2}}.
\end{split}
\end{equation}

It's easy to check that these states are all mutually orthogonal.
Remember that $\ket{00}$ is shorthand for $\ket{0}\otimes \ket{0}$ and that the inner product on the tensor product is
\[
	\big(a\ket{v_1}\otimes \ket{w_1}, b\ket{v_2}\otimes \ket{w_2}\big) = a^*b \langle v_1|v_2\rangle \langle w_1 | w_2\rangle.
\]
\end{document}