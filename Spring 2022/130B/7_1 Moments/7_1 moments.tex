\documentclass[11pt,letterpaper]{article}
\usepackage{amssymb,amsfonts,color,graphicx,amsmath,enumerate}
\usepackage{tikz}
\usepackage{amsthm}

\newcommand{\naturals}{\mathbb{N}}
\newcommand{\integers}{\mathbb{Z}}
\newcommand{\complex}{\mathbb{C}}
\newcommand{\reals}{\mathbb{R}}
\newcommand{\mcal}[1]{\mathcal{#1}}
\newcommand{\rationals}{\mathbb{Q}}
\newcommand{\Lp}[2]{\left\|{#1}\right\|_{L^{#2}}}
\newcommand{\field}{\mathbb{F}}
\newcommand{\affine}{\mathbb{A}}
\newcommand{\E}{\mathbb{E}}
\newcommand{\Prob}{\mathbb{P}}
\newcommand{\Var}{\text{Var}}
\newcommand{\ind}{\mathbbm{1}}
\newcommand{\Cov}{\text{Cov}}

\newenvironment{solution}
{\begin{proof}[Solution]}
{\end{proof}}

\voffset=-3cm
\hoffset=-2.25cm
\textheight=24cm
\textwidth=17.25cm
\addtolength{\jot}{8pt}
\linespread{1.3}

\begin{document}
\begin{center}
{\bf \Large Math 130B - Variance, Covariance}
\vspace{0.2cm}
\hrule
\end{center}

\begin{enumerate}
	\item If $X$ and $Y$ are independent and identically distributed with mean $\mu$ and variance $\sigma^2$, find
	\[
		\E[(X-Y)^2].
	\]

	\vfill

	\item Show that $\E[(X-a)^2]$ is minimized when $a = \E[X]$.

	\vfill

	\item Let $X_1, \ldots, X_n$ be iid continuous random variables. We say that a record value occurs at time $i$, $i\leq n$ if $X_i \geq X_k$ for all $k\leq i$. Show that
	\begin{enumerate}
		\item $\E[$number of record values$] = \sum_{i=1}^n 1/i$.
		\vfill

		\item $\Var[$number of record values$] = \sum_{i=1}^n (i-1)/i^2$.
	\end{enumerate}
	\vfill

	\item Suppose that $X$ and $Y$ are identically distributed, but not necessarily independent. Show that $X+Y$ and $X-Y$ are uncorrelated.

	\vfill

	\item A multilevel marketing firm operates as follows. Person 1 starts the firm, then recruits person 2. Persons 1 and 2 then compete to recruit person 3 (who is always recruited in the end). Then persons 1, 2 and 3 compete to recruit person 4, and so on. Suppose that when persons 1 through $i$ compete to recruit person $i+1$, they are all equally likely to succeed (but one of them for sure succeeds). This goes on until $n$ people work at the firm.
	\begin{enumerate}
		\item Find the expected number of people $1, \ldots, n$ who did not recruit anyone else.

		\vfill

		\item Come up with an expression for the variance of the number of people who don't recruit anyone.
	\end{enumerate}

	\vfill
\end{enumerate}

\end{document}