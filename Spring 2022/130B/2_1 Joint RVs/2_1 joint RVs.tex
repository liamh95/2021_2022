\documentclass[11pt,letterpaper]{article}
\usepackage{amssymb,amsfonts,color,graphicx,amsmath,enumerate}
\usepackage{tikz}
\usepackage{amsthm}

\newcommand{\naturals}{\mathbb{N}}
\newcommand{\integers}{\mathbb{Z}}
\newcommand{\complex}{\mathbb{C}}
\newcommand{\reals}{\mathbb{R}}
\newcommand{\mcal}[1]{\mathcal{#1}}
\newcommand{\rationals}{\mathbb{Q}}
\newcommand{\Lp}[2]{\left\|{#1}\right\|_{L^{#2}}}
\newcommand{\field}{\mathbb{F}}
\newcommand{\affine}{\mathbb{A}}
\newcommand{\E}{\mathbb{E}}
\newcommand{\Prob}{\mathbb{P}}
\newcommand{\Var}{\text{Var}}
\newcommand{\ind}{\mathbbm{1}}
\newcommand{\Cov}{\text{Cov}}

\newenvironment{solution}
{\begin{proof}[Solution]}
{\end{proof}}

\voffset=-3cm
\hoffset=-2.25cm
\textheight=24cm
\textwidth=17.25cm
\addtolength{\jot}{8pt}
\linespread{1.3}

\begin{document}
\begin{center}
{\bf \Large Math 130B - Joint Random Variables}
\vspace{0.2cm}
\hrule
\end{center}

%\textbf{Do the fourth problem in each section.}
\begin{enumerate}
	\item Two fair dice are rolled. Fined the joint probability mass function of $X$ and $Y$ when
	\begin{enumerate}
		\item $X$ is the value on the first die and $Y$ is the larger of the two values.
		\item $X$ is the smallest and $Y$ is the largest value obtained on the dice.
	\end{enumerate}

	\vfill

	\item The joint density of $X,Y$ is given by
	\[
		f(x,y) = \begin{cases}
			3x &\text{if }0\leq y \leq x \leq 1\\
			0 & \text{otherwise}
		\end{cases}.
	\]
	\begin{enumerate}
		\item Find the marginal densities $f_X$ and $f_Y$.

		\item Compute $Cov(X,Y)$.
	\end{enumerate}

	\vfill

	\item Let $X$ and $Y$ be independent random variables taking values in the positive integers having the same distribution given by
	\[
		\Pr[X=n] = \Pr[Y=n] = 2^{-n}
	\]
	for all $n \geq 1$. Find $\Pr[X \text{ divides }Y]$.

	\vfill

	\item Let $X$ and $Y$ be independent continuous random variables with densities $f_X$ and $f_Y$, respectively. Express the density of $XY$ in terms of the densities of $X$ and $Y$.

	\vfill

	\item Suppose that $n$ points are independently chosen at random on the circumference of a circle, and we want the probability that they all lie in a semicircle. That is, we want the probability that there is a line passing through the center of the circle such that all the points are on the same side of that line.

	Let $P_1 \ldots, P_n$ be the $n$ points. Let $A^{(n)}$ denote the event that all the points are contained in some semicircle, and let $A^{(n)}_i$ be the event that all the points lie in the semicircle beginning at the point $P_i$ and going clockwise $180^\circ$ for $i = 1, \ldots, n$.
	\begin{enumerate}
		\item Express $A^{(n)}$ in terms of the $A^{(n)}_i$.
		\item Find $\Pr[A^{(n)}]$ and describe what happens to this as $n\to \infty$.
	\end{enumerate}
	\vfill


\end{enumerate}

\end{document}