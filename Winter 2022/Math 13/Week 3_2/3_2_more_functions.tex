\documentclass[12pt]{report}  

\pagestyle{empty}

\usepackage{graphics}
\usepackage{amsmath,amssymb,amsthm, multicol,array}
\usepackage[pdftex]{graphicx}
\usepackage{enumerate}
\usepackage{epsf}

\theoremstyle{definition}
\newtheorem{thm}{Theorem}
\newtheorem{lem}[thm]{Lemma}
\newtheorem{cor}[thm]{Corollary}
\newtheorem{rem}[thm]{Remark}
\newtheorem{remark}[thm]{Remark}
\newtheorem{conj}[thm]{Conjecture}
\newtheorem{definition}[thm]{Definition}

\newcommand{\naturals}{\mathbb{N}}
\newcommand{\integers}{\mathbb{Z}}
\newcommand{\complex}{\mathbb{C}}
\newcommand{\reals}{\mathbb{R}}
\newcommand{\mcal}[1]{\mathcal{#1}}
\newcommand{\rationals}{\mathbb{Q}}
\newcommand{\Aut}{\text{Aut}}
\newcommand{\Lp}[2]{\left\|{#1}\right\|_{L^{#2}}}
\newcommand{\tr}{\text{tr}}
\newcommand{\field}{\mathbb{F}}

\addtolength{\oddsidemargin}{-.75in}
\addtolength{\evensidemargin}{-.75in}
\addtolength{\textwidth}{1.5in}
\addtolength{\topmargin}{-1in}
\addtolength{\textheight}{2.25in}

\begin{document}
\begin{center}
{\bf \Large Math 13 - Week 3: More on Functions}
\vspace{0.2cm}
\hrule
\end{center}

\begin{enumerate}	

	% \item Which of these real life scenarios can be (easily) described by a function? If a function is appropriate, is it one-to-one, onto, or bijective?
	% \begin{enumerate}
	% 	\item a
	% \end{enumerate}

	%\vfill

	\item Let $A$ and $B$ be finite sets. Prove that the following are equivalent (that is, prove that if any one of these statements is true, then the others are as well).
	\begin{enumerate}
		\item If $|A|\leq |B|$.
		\item There is a one-to-one function $f: A\to B$.
		\item There is an onto function $g: B\to A$.
	\end{enumerate}

	\vfill

	\item Use the last exercise to show that if $A$ and $B$ are finite sets, then $|A| = |B|$ if and only if there is a bijection $f: A\to B$.

	\vfill

	\item Let $f: A\to B$ and $g: B\to C$ be functions. Prove that:
	\begin{enumerate}
		\item If $f$ and $g$ are one-to-one, then so is $g\circ f$.
		\item If $f$ and $g$ are onto, then so is $g\circ f$.
		\item If $g\circ f$ is one-to-one, then so is $f$.
		\item If $g\circ f$ is onto, then so is $g$.
	\end{enumerate}

	\vfill

\end{enumerate}

\end{document}