\documentclass[12pt]{report}  

\pagestyle{empty}

\usepackage{graphics}
\usepackage{amsmath,amssymb,amsthm, multicol,array}
\usepackage[pdftex]{graphicx}
\usepackage{enumerate}
\usepackage{epsf}

\theoremstyle{definition}
\newtheorem{thm}{Theorem}
\newtheorem{lem}[thm]{Lemma}
\newtheorem{cor}[thm]{Corollary}
\newtheorem{rem}[thm]{Remark}
\newtheorem{remark}[thm]{Remark}
\newtheorem{conj}[thm]{Conjecture}
\newtheorem{definition}[thm]{Definition}

\newcommand{\naturals}{\mathbb{N}}
\newcommand{\integers}{\mathbb{Z}}
\newcommand{\complex}{\mathbb{C}}
\newcommand{\reals}{\mathbb{R}}
\newcommand{\mcal}[1]{\mathcal{#1}}
\newcommand{\rationals}{\mathbb{Q}}
\newcommand{\Aut}{\text{Aut}}
\newcommand{\Lp}[2]{\left\|{#1}\right\|_{L^{#2}}}
\newcommand{\tr}{\text{tr}}
\newcommand{\field}{\mathbb{F}}

\addtolength{\oddsidemargin}{-.75in}
\addtolength{\evensidemargin}{-.75in}
\addtolength{\textwidth}{1.5in}
\addtolength{\topmargin}{-1in}
\addtolength{\textheight}{2.25in}

\begin{document}
\begin{center}
{\bf \Large Math 13 - Week 7: Number Theory}
\vspace{0.2cm}
\hrule
\end{center}

\begin{enumerate}

\item Prove that $\sqrt{3}$ is irrational.

\vfill

\item Solve the following equations.
\begin{enumerate}
	\item $3x \equiv 4$ in $\integers_{11}$.
	\item $4x-8 \equiv 9$ in $\integers_{11}$.
	\item $3x+8 \equiv 1$ in $\integers_{10}$.
\end{enumerate}

\vfill

\item Solve the following equations. How are these problems different from the previous ones. Find \textit{all} solutions.
\begin{enumerate}
	\item $2x \equiv 4$ in $\integers_{10}$.
	\item $2x \equiv 3$ in $\integers_{10}$.
	\item $9x \equiv 4$ in $\integers_{12}$.
	\item $9x \equiv 6$ in $\integers_{12}$.
\end{enumerate}

\vfill

\item Let $x$ be an integer. Prove that $2\mid x$ and $3\mid x$ if and only if $6\mid x$.

\vfill

\item Let $a$ and $b$ be positive integers. Prove that $a$ and $b$ are relatively prime if and only if there is no prime $p$ such that $p\mid a$ and $p\mid b$.

\vfill

% \item Let $a,b\in \integers$ with $a\geq b > 0$.
% In this exercise we explore how many steps it takes to compute the greatest common divisor of $a$ and $b$ using the Euclidean algorithm.
% \begin{enumerate}
% 	\item Let $c = a\mod b$. Show that $c < a/2$. \textit{Hint: consider two cases -- one where $a<2b$ and one where $a \geq 2b$.}

% 	\item Recall that the Euclidean algorithm produces a sequence of numbers
% 	\[
% 		r_0 \geq r_1 \geq r_2 \geq \cdots \geq 0,
% 	\]
% 	where $r_0 = a$, $r_1 = b$ and $r_{i+2} = r_i \mod r_{i+1}$.
% 	Using the previous exercise, conclude that for any $i$, the numbers $r_{i+2}$ and $r_{i+3}$ are less than half as large as $r_i$ and $r_{i+1}$, respectively.
% 	In other words, show that after $2t$ steps, the numbers we're working with drop by more than a factor of $2^t$.

% 	\item The Euclidean algorithm terminates as soon as we reach a remainder of zero. If $T$ is the last step of the algorithm, conclude that
% 	\[
% 		2^{-T}b\leq 1.
% 	\]
% 	Solve this for $T$ to obtain a bound for how many steps it takes to run the Euclidean algorithm.
% \end{enumerate}

% \vfill

\end{enumerate}

\end{document}