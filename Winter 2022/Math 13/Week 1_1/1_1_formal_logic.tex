\documentclass[12pt]{report}  

\pagestyle{empty}

\usepackage{graphics}
\usepackage{amsmath,amssymb,amsthm, multicol,array}
\usepackage[pdftex]{graphicx}
\usepackage{enumerate}
\usepackage{epsf}

\theoremstyle{definition}
\newtheorem{thm}{Theorem}
\newtheorem{lem}[thm]{Lemma}
\newtheorem{cor}[thm]{Corollary}
\newtheorem{rem}[thm]{Remark}
\newtheorem{remark}[thm]{Remark}
\newtheorem{conj}[thm]{Conjecture}
\newtheorem{definition}[thm]{Definition}

\newcommand{\naturals}{\mathbb{N}}
\newcommand{\integers}{\mathbb{Z}}
\newcommand{\complex}{\mathbb{C}}
\newcommand{\reals}{\mathbb{R}}
\newcommand{\mcal}[1]{\mathcal{#1}}
\newcommand{\rationals}{\mathbb{Q}}
\newcommand{\Aut}{\text{Aut}}
\newcommand{\Lp}[2]{\left\|{#1}\right\|_{L^{#2}}}
\newcommand{\tr}{\text{tr}}
\newcommand{\field}{\mathbb{F}}

\addtolength{\oddsidemargin}{-.75in}
\addtolength{\evensidemargin}{-.75in}
\addtolength{\textwidth}{1.5in}
\addtolength{\topmargin}{-1in}
\addtolength{\textheight}{2.25in}

\begin{document}
\begin{center}
{\bf \Large Math 13 - Week 1: Formal Logic}
\vspace{0.2cm}
\hrule
\end{center}

\begin{definition}
	A function $f: \{0, 1\}^k\to \{0, 1\}$ is called a \textbf{Boolean function with arity} $k$.
	This notation means that the input of $f$ is a sequence of $k$ 0's and 1's and its output is a single 0 or 1 (think of $0$ as ``FALSE'' and $1$ as ``TRUE''). If $k$ is small, then we sometimes refer to its variables as $P$, $Q$ or $R$. If $k$ is larger, then we refer to its variables as $x_1, x_2, \ldots, x_k$.
\end{definition}

We can write down Boolean functions in a few ways.
One that you've seen in lecture is the \textbf{truth table}, which has one column for each input variable and one column for the output.
In any row, the value in the output column is the result of applying the function to the corresponding input values.
For example, here's the truth table for the Boolean function OR.
\begin{center}
	\begin{tabular}{|c|c||c|}
		\hline
		$P$ & $Q$ & $P$ OR $Q$\\
		\hline
		\hline
		0 & 0 & 0\\
		\hline
		0 & 1 & 1\\
		\hline
		1 & 0 & 1\\
		\hline
		1 & 1 & 1\\
		\hline
	\end{tabular}
\end{center}

\begin{enumerate}
	\item Write the truth table for the Boolean function $Maj_3$, which takes 3 arguments and returns true if at least two of its arguments are true and returns false if at least two of its arguments are false.

	\vfill

	\item Use your truth table from the previous problem to write $Maj_3$ using only $\land$, $\lor$ and $\neg$. Use the same idea to write $\implies$ using only $\land$, $\lor$ and $\neg$.

	\vfill

	\item Use the reasoning from the previous problem to argue that \emph{any} Boolean function can be written using $\land$, $\lor$ and $\neg$.

	\vfill 

	\pagebreak

	\item Let $A = \{1, 2. \{3, 4\}\}$. Which of the following are true and which are false?
	\begin{enumerate}
		\item $1\in A$.
		\item $\{1\}\in A$.
		\item $3\in A$.
		\item $\{3\}\in A$.
		\item $\{3\}\subseteq A$.
	\end{enumerate}

	\vfill

	\item Let $A$, $B$ and $C$ be sets. Prove that $(A\cup B)\setminus C = (A\setminus C)\cup (B\setminus C)$.
	\vfill

	\item Can you write the operation of $\cap$ using only the operations $\cup$ and $\setminus$?
	\vfill

\end{enumerate}

\end{document}