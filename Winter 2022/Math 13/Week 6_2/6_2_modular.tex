\documentclass[12pt]{report}  

\pagestyle{empty}

\usepackage{graphics}
\usepackage{amsmath,amssymb,amsthm, multicol,array}
\usepackage[pdftex]{graphicx}
\usepackage{enumerate}
\usepackage{epsf}

\theoremstyle{definition}
\newtheorem{thm}{Theorem}
\newtheorem{lem}[thm]{Lemma}
\newtheorem{cor}[thm]{Corollary}
\newtheorem{rem}[thm]{Remark}
\newtheorem{remark}[thm]{Remark}
\newtheorem{conj}[thm]{Conjecture}
\newtheorem{definition}[thm]{Definition}

\newcommand{\naturals}{\mathbb{N}}
\newcommand{\integers}{\mathbb{Z}}
\newcommand{\complex}{\mathbb{C}}
\newcommand{\reals}{\mathbb{R}}
\newcommand{\mcal}[1]{\mathcal{#1}}
\newcommand{\rationals}{\mathbb{Q}}
\newcommand{\Aut}{\text{Aut}}
\newcommand{\Lp}[2]{\left\|{#1}\right\|_{L^{#2}}}
\newcommand{\tr}{\text{tr}}
\newcommand{\field}{\mathbb{F}}

\addtolength{\oddsidemargin}{-.75in}
\addtolength{\evensidemargin}{-.75in}
\addtolength{\textwidth}{1.5in}
\addtolength{\topmargin}{-1in}
\addtolength{\textheight}{2.25in}

\begin{document}
\begin{center}
{\bf \Large Math 13 - Week 6: Modular Arithmetic}
\vspace{0.2cm}
\hrule
\end{center}

\begin{enumerate}

\item Which of these assignments are injective, surjective or bijective?
\begin{enumerate}
	\item In 2016, UCI had 33,467 students enrolled. Let $f$ be the function that maps each student to their eight-digit student ID number. (Consider $f$ as a function from the set of students to the set of all eight-digit numbers).

	\vfill
	\item Let $g$ be the function from the set of ID numbers of current UCI students to the set of current UCI students that maps an ID number to the student it belongs to.

	\vfill

	\item A padlock company produces 100,000 padlocks in a month. Each padlock is opened with a combination of three numbers, each between 0 and 39 and each lock is made with a random combination. Leg $h$ be the function that maps each padlock to the three-number sequence representing its combination.

\end{enumerate}

\vfill

\item Find $x$ and $y$ such that $431x + 29y = \gcd(431, 29)$. \textit{I promise it won't take that many steps.}

\vfill

\item Prove that consecutive integers must be relatively prime (that is, their greatest common divisor is 1).

\vfill

\item Let $a$ be an integer. Prove that $2a+1$ and $4a^2+1$ are relatively prime.

\vfill

\item Suppose that $a$ and $b$ are relatively prime integers and that $a\mid c$ and $b\mid c$. Prove that $(ab)\mid c$.
\vfill

% \item Let $a,b\in \integers$ with $a\geq b > 0$.
% In this exercise we explore how many steps it takes to compute the greatest common divisor of $a$ and $b$ using the Euclidean algorithm.
% \begin{enumerate}
% 	\item Let $c = a\mod b$. Show that $c < a/2$. \textit{Hint: consider two cases -- one where $a<2b$ and one where $a \geq 2b$.}

% 	\item Recall that the Euclidean algorithm produces a sequence of numbers
% 	\[
% 		r_0 \geq r_1 \geq r_2 \geq \cdots \geq 0,
% 	\]
% 	where $r_0 = a$, $r_1 = b$ and $r_{i+2} = r_i \mod r_{i+1}$.
% 	Using the previous exercise, conclude that for any $i$, the numbers $r_{i+2}$ and $r_{i+3}$ are less than half as large as $r_i$ and $r_{i+1}$, respectively.
% 	In other words, show that after $2t$ steps, the numbers we're working with drop by more than a factor of $2^t$.

% 	\item The Euclidean algorithm terminates as soon as we reach a remainder of zero. If $T$ is the last step of the algorithm, conclude that
% 	\[
% 		2^{-T}b\leq 1.
% 	\]
% 	Solve this for $T$ to obtain a bound for how many steps it takes to run the Euclidean algorithm.
% \end{enumerate}

% \vfill

\end{enumerate}

\end{document}