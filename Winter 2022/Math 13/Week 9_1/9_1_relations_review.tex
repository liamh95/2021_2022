\documentclass[12pt]{report}  

\pagestyle{empty}

\usepackage{graphics}
\usepackage{amsmath,amssymb,amsthm, multicol,array}
\usepackage[pdftex]{graphicx}
\usepackage{enumerate}
\usepackage{epsf}

\theoremstyle{definition}
\newtheorem{thm}{Theorem}
\newtheorem{claim}{Claim}
\newtheorem{lem}[thm]{Lemma}
\newtheorem{cor}[thm]{Corollary}
\newtheorem{rem}[thm]{Remark}
\newtheorem{remark}[thm]{Remark}
\newtheorem{conj}[thm]{Conjecture}
\newtheorem{definition}[thm]{Definition}

\newcommand{\naturals}{\mathbb{N}}
\newcommand{\integers}{\mathbb{Z}}
\newcommand{\complex}{\mathbb{C}}
\newcommand{\reals}{\mathbb{R}}
\newcommand{\mcal}[1]{\mathcal{#1}}
\newcommand{\rationals}{\mathbb{Q}}
\newcommand{\Aut}{\text{Aut}}
\newcommand{\Lp}[2]{\left\|{#1}\right\|_{L^{#2}}}
\newcommand{\tr}{\text{tr}}
\newcommand{\field}{\mathbb{F}}

\addtolength{\oddsidemargin}{-.75in}
\addtolength{\evensidemargin}{-.75in}
\addtolength{\textwidth}{1.5in}
\addtolength{\topmargin}{-1in}
\addtolength{\textheight}{2.25in}

\begin{document}
\begin{center}
{\bf \Large Math 13 - Week 9: Equivalence Relations}
\vspace{0.2cm}
\hrule
\end{center}

\begin{enumerate}

\item Let $f: \reals\to \reals$ be defined by $f(t) = (-t, t^3)$ and let $g: \reals^2\to \reals$ be given by $g(x,y) = \cos(x^2+y^2)$.
Are these claims valid? If not, explain the error.
\begin{enumerate}
	\item The function $g$ surjects onto $\reals$ because $\reals^2$ is a higher dimensional space than $\reals$.

	\vfill

	\item The function $f$ surjects onto $\reals^2$. To see this, take $(-t, t^3)\in \reals^2$. Then $(-t, t^3) = f(t)$, so $f$ is surjective.

	\vfill

	\item The function $f$ surjects because for every $b$ in $B$ there is an $a$ in $A$ such that $f(a) = b$.

	\vfill

	\item $g$ is injective because $g(x,y) = g(x',y')$ implies that $(x,y) = (x',y')$.

	\vfill

	\item $f$ is injective because the only way for $f$ to hit $(-1, 1)$ is to plug in $t=1$.
	\vfill
\end{enumerate}

\item Let $\sim$ be the relation on $\reals^2$ defined by $(x,y)\sim (v,w)$ if and only if $x^2+y^2 = v^2+w^2$.
\begin{enumerate}
	\item Prove that $\sim$ is an equivalence relation.

	\vfill

	\item Describe the quotient $\reals^2 / \sim$. Think geometrically.
\end{enumerate}

\vfill

\item Consider the following claim and proof.
\begin{claim}
	If a relation is symmetric and transitive, then it is reflexive.
\end{claim}
\begin{proof}
	Suppose $\sim$ is a relation on a set $S$ that is symmetric and transitive.
	Then for any $x,y$ in $S$, $x\sim y$ implies $y\sim x$ by symmetry.
	Since $x\sim y$ and $y\sim x$, then $x\sim x$ by transitivity and the relation is reflexive.
\end{proof}
Is this proof valid? Why or why not?

\vfill
\end{enumerate}

\end{document}